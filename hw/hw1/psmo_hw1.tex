\documentclass[10pt, a4paper]{extarticle}

%% Язык
\usepackage{cmap} % Поиск в PDF
\usepackage{mathtext} % Кириллица в формулах
\usepackage[T2A]{fontenc} % Кодировка
\usepackage[utf8]{inputenc} % Кодировка
\usepackage[english,russian]{babel} % Локализация, переносы

%% Шрифты

% Serif
%\usepackage{euscript} % Шрифт Евклид
%\usepackage{mathrsfs} % Шрифт для математики
\usepackage{libertinus}

% Sans-serif
%\renewcommand{\rmdefault}{cmss}
%\renewcommand{\ttdefault}{cmss}
%\usepackage{sfmath}

% Настройки для xelatex
%\usepackage{polyglossia} % Для выбора языка в xelatex
%\setmainlanguage{russian}
%\setotherlanguages{english}
% Ligatures=TeX is on by default
% https://tex.stackexchange.com/questions/323542/
%\setmainfont[Ligatures=TeX]{Cantarell}
%\newfontfamily{\cyrillicfonttt}{Times New Roman}
%\newfontfamily\cyrillicfont{Cantarell}[Script=Cyrillic]
%\setsansfont[Ligatures=TeX]{Cantarell}
%\newfontfamily\cyrillicfontsf{Cantarell}[Script=Cyrillic]
%\setmonofont{Courier New}
%\newfontfamily\cyrillicfonttt{Courier New}[Script=Cyrillic]

%% Математика
\usepackage{amsmath, amsfonts, amssymb, amsthm, mathtools}
\usepackage{icomma}

% Операторы
\DeclareMathOperator*\plim{plim}
\DeclareMathOperator{\sgn}{sign}
\DeclareMathOperator{\sign}{sign}
\DeclareMathOperator*{\argmin}{arg\,min}
\DeclareMathOperator*{\argmax}{arg\,max}
\DeclareMathOperator*{\amn}{arg\,min}
\DeclareMathOperator*{\amx}{arg\,max}
\DeclareMathOperator{\cov}{Cov}
\DeclareMathOperator{\Var}{Var}
\DeclareMathOperator{\Cov}{Cov}
\DeclareMathOperator{\Corr}{Corr}
\DeclareMathOperator{\pCorr}{pCorr}
\DeclareMathOperator{\E}{\mathbb{E}}
\let\P\relax
\DeclareMathOperator{\P}{\mathbb{P}}
\renewcommand{\le}{\leqslant}
\renewcommand{\ge}{\geqslant}
\renewcommand{\leq}{\leqslant}
\renewcommand{\geq}{\geqslant}

% Распределения
\newcommand{\cN}{\mathcal{N}}
\newcommand{\cU}{\mathcal{U}}
\newcommand{\cBinom}{\mathcal{Binom}}
\newcommand{\cPois}{\mathcal{Pois}}
\newcommand{\cBeta}{\mathcal{Beta}}
\newcommand{\cGamma}{\mathcal{Gamma}}

% Множества
\def \R{\mathbb{R}}
\def \N{\mathbb{N}}
\def \Z{\mathbb{Z}}

% Другое
\newcommand{\dx}[1]{\,\mathrm{d}#1} % Для интеграла: маленький отступ и прямая d
\newcommand{\ind}[1]{\mathbbm{1}_{\{#1\}}} % Индикатор события
\newcommand{\iid}{\mathrel{\stackrel{\rm i.\,i.\,d.}\sim}}
\newcommand{\const}{\mathrm{const}}

%% Изображения
\usepackage{graphicx}
\usepackage{caption}
\usepackage{subcaption}
\usepackage{physics}
\usepackage{wrapfig} % Обтекание рисунков и таблиц текстом
\usepackage{tikz}

%% Таблицы
\usepackage{array, tabularx, tabulary, booktabs}
\usepackage{longtable}  % Длинные таблицы
\usepackage{multirow} % Слияние строк в таблице

%% Cписки
\usepackage{multicol}
\usepackage{enumitem}

%% Гиперссылки
\usepackage{xcolor}
\usepackage{hyperref}
\definecolor{linkcolor}{HTML}{8b00ff}
\hypersetup{colorlinks = true,
			linkcolor = linkcolor,
			urlcolor = linkcolor,
			citecolor = linkcolor}

%% Выравнивание
\setlength{\parskip}{0.5em} % Расстояние между абзацами
\usepackage{geometry} % Поля
\geometry{
	a4paper,
	left=20mm,
	top=20mm,
	right=20mm}
\setlength{\parindent}{0cm} % Отступ (красная строка)
\linespread{1.0} % Интерлиньяж
\usepackage[many]{tcolorbox}  

%% Оформление

\newtcolorbox{rulesbox}[1]{%
	tikznode boxed title,
	enhanced,
	arc=0mm,
	interior style={white},
	attach boxed title to top center= {yshift=-\tcboxedtitleheight/2},
	fonttitle=\bfseries,
	colbacktitle=white,coltitle=black,
	boxed title style={size=normal,colframe=white,boxrule=0pt},
	title={#1}}

% Красивый серый фон
\usepackage{framed} 
\definecolor{shadecolor}{gray}{0.9}

% Код
\newcommand{\code}[1]{{\tt #1}}

% Колонтитулы
\usepackage{fancyhdr}
\pagestyle{fancy}
\fancyhf{}
\fancyhead[L]{}
\fancyhead[R]{\thepage}

% Разделы и подразделы
\usepackage[sf, sl, outermarks]{titlesec}
\titleformat{\section}{\Large\bfseries\sffamily}{\thesection}{0.5em}{}
\titleformat{\subsection}{\large\sffamily}{\thesubsection}{0.5em}{}

% Содержание
%\usepackage{tocloft}
%\renewcommand{\cftsecfont}{\hspace{4.5em}\normalfont}
%\renewcommand{\cftsubsecfont}{\hspace{5em}\normalfont}
%\renewcommand{\cftsecpagefont}{\normalfont\hfill}
%\renewcommand{\cfttoctitlefont}{\large\normalfont\hfill}
%\renewcommand{\cftaftertoctitle}{\hfill}
%\renewcommand{\cftsecleader}{\cftdotfill{\cftdotsep}}
%\renewcommand{\cftsecafterpnum}{\hspace*{5.5em}\hfill}
%\renewcommand{\cftsubsecafterpnum}{\hspace*{5.5em}\hfill}
%\renewcommand{\cftsecaftersnum}{.}
%\renewcommand{\cftsubsecaftersnum}{.}

%% Комментарии
\usepackage{comment}

%% To-do
\usepackage{todonotes}

%% Литература
\usepackage[backend = biber,
			bibencoding = utf8, 
			sorting = nty, 
			maxcitenames = 4,
			style = numeric-verb]{biblatex}
\addbibresource{lit.bib}
\usepackage{csquotes}

%% Заголовок
\title{{\normalsize Прикладная статистика в машинном обучении} \\\vspace{0.5em} Домашнее задание \#1}
\author{Дедлайн: 11 октября, 23:59 МСК}
\date{\rule{15cm}{0.4pt}}




\begin{document}
	
	\maketitle
	
	\begin{rulesbox}{Правила игры}
		\begin{enumerate}
			\item Домашнее задание оценивается в $10$ баллов.
			\item Решения принимаются до \textbf{11 октября 2020 года, 23:59 МСК} включительно.
			Работы, отправленные после дедлайна, оцениваются следующим образом:
			\begin{itemize}
				\item До 7:00 МСК 12 октября: максимум 8 баллов.
				\item До 23:59 МСК 12 октября: максимум 7 баллов.
				\item До 23:59 МСК 13 октября: максимум 6 баллов.
			\end{itemize}
			Работы, отправленные после 13-го октября, будут проверены без оценки.
			\item Все решения нужно загрузить в личный репозиторий на \href{https://classroom.github.com/a/szspisNo}{GitHub Classroom}.
			\item Репозиторий должен содержать:
				\begin{description}
					\item[либо:] PDF-файл с решениями теоретических задач и .ipynb-файл с решениями экспериментальных задач.
					Решение теоретических задач можно набрать в любом электронном редакторе или написать от руки, а затем сделать качественный скан.
					Все решения должны быть расположены в правильном порядке в одном файле.
					Если экспериментальная задача является частью теоретической, то в .ipynb-файле нужно явно указать номер теоретической задачи.
					
					PDF-файл должен иметь название name\_surname\_hw1.pdf, а .ipynb-файл должен называться name\_surname\_hw1\_code.pdf
					\item[либо:] один .ipynb-файл с решениями и практических, и теоретических задач, оформленных в ячейках Markdown.
					
					.ipynb-файл должен иметь название name\_surname\_hw1.pdf
				\end{description}
			 \item Весь код должен быть написан на Python.
			 \item Разрешается использовать без доказательства любые результаты, встречавшиеся на лекциях или семинарах по курсу, если получение этих результатов не является вопросом задания. 
			 \item Разрешается использовать любые свободные источники с указанием ссылки на них.
			 \item Плагиат не допускается.
			 При обнаружении случаев списывания, $0$ за работу выставляется всем участникам нарушения, даже если можно установить, кто у кого списал. 
		\end{enumerate}
	\end{rulesbox}
	
	\newpage

	{\Large \textbf{Задача 1.} Полезное утверждение (3 балла)}
	
	Гарри никак не может понять, почему при большой информации Фишера оценки максимального правдоподобия лежат к истинному параметру ближе, чем при малой информации Фишера.
	Гермиона решает продемонстрировать аналитическую интуицию, стоящую за этим утверждением:
	
	<<Если взять выборку независимых одинаково распределённых случайных величин $Y_1$, $\ldots$, $Y_N$, каждая из которых имеет функцию плотности или функцию вероятности $f(y | \theta)$, и предположить, что выполнены все необходимые условия регулярности, то при $\phi \to \theta$:
	\begin{align*}
	D_{KL}\left[f(y | \theta), f(y | \phi)\right] = \dfrac{1}{2} I(\theta)(\phi - \theta)^2 + O((\phi - \theta)^3)\text{>>}.
	\end{align*}
	
	\begin{enumerate}[label=\textbf{\alph*)}]
		\item Докажите утверждение Гермионы либо для случая функций плотности, либо для случая функций вероятности.
		\item Поясните Гарри, почему при большей информации Фишера ML-оценки лежат ближе к истинному параметру. 
	\end{enumerate}

	\textit{Подсказка:} $H(f) = \E(\ln f)$, аналогично для кросс-энтропии.
	
	\begin{flushright}
		\textit{\small (По мотивам: Williams, Weighing the Odds)}
	\end{flushright}

	
	{\Large \textbf{Задача 2.} Непростительные заклинания (2 балла)}
	
	Рон убеждён, что время, которое уходит на произнесение одного непростительного заклинания -- это непрерывная случайная величина с плотностью
	\[
	f(x | q) = \begin{cases}
		\frac{2x}{q} e^{\frac{-x^2}{q}}, &\text{ если } x > 0, \\
		0, &\text{ иначе,}
	\end{cases}
	\]
	где $q > 0$.
	Предположим, что все непростительные заклинания произносятся за одинаковое время.
	Рон длительное время наблюдал за тёмными волшебниками, а потому собрал случайную выборку $X_1$, $\ldots$ $X_N$, где $X_i$ -- время произнесения непростительного заклинания, а $N$ очень велико.
	
	\begin{enumerate}[label=\textbf{\alph*)}]
		\item Найдите $\hat{q}_{ML}$.
		\item Найдите $\widehat{(q^2)}_{ML}$
		\item Постройте 95\%-ый доверительный интервал для $q$.
		\item Повторите предыдущие пункты на компьютере, численно оптимизировав функцию правдоподобия.
		Подробно опишите, с какими трудностями вы столкнулись в процессе и как вы их преодолели.
		Если трудностей не возникло, также напишите об этом. 
	\end{enumerate}

	{\Large \textbf{Задача 3.} Палочки и волшебники (2 балла)}
	
	Дамблдор уверен, что среди первокурсников встречаются только обладатели палочек из вишни, дуба и вяза.
	Воспользовавшись своими способностями в легилименции на пиру в честь начала нового учебного года, Дамблдор узнаёт, что из $150$ первокурсников $75$ имеют палочки из вишни, $30$ -- из дуба и $45$ -- из вяза.
	Дамблдор считает, что палочки выбирают волшебников независимо друг от друга, и вероятность того, что у волшебника окажется вишнёвая палочка, равна $p_1$, а что дубовая, равна $p_2$.
	
	\begin{enumerate}[label=\textbf{\alph*)}]
		\item Обозначим $p = \begin{pmatrix}
			p_1 \\
			p_2
		\end{pmatrix}$. Найдите $\hat{p}_{ML}$.
		
		\item Проверьте гипотезу $H_0: p_1 = 0.7$ против $H_A: p_1 \ne 0.7$ на уровне значимости $5\%$ при помощи тестов $LR$ и $LM$.
		
		\item Проверьте гипотезу
		\[
		H_0: \begin{pmatrix}
			p_1 \\
			p_2
		\end{pmatrix} = \begin{pmatrix}
		0.4 \\
		0.6
	\end{pmatrix}
	\]
 	против 
 	\[
 	H_A: \begin{pmatrix}
 		p_1 \\
 		p_2
 	\end{pmatrix} \ne \begin{pmatrix}
 		0.4 \\
 		0.6
 	\end{pmatrix}
 	\]
 	на уровне значимости $5\%$ при помощи теста $W$.
 	\item Постройте $95\%$-ый доверительный интервал для $p_1 - p_2$.
	\end{enumerate}

	
	{\Large \textbf{Задача 4.} Модель для зелий (3 балла)}
	
	Полумна хочет построить предсказательную модель, которая бы описывала зависимость популярности зелья $y_i$ от силы его положительного влияния $x_i$.
	Обе величины являются количественными непрерывными переменными на $\R$.
	Предположим, что Полумна знает, как измерить популярность и силу влияния и верит, что искомая зависимость имеет следующий вид:
	\[
	y_i = \beta_1 e^{\beta_2 x_i}u_i,
	\]
	где $\beta_1$ и $\beta_2$ -- неизвестные коэффициенты, не равные нулю, $u_i $ -- случайная ошибка, причём $\ln u_i \sim \mathcal{N}(0,1)$.
	
	\begin{enumerate}[label=\textbf{\alph*)}]
		\item Является ли данная зависимость линейной по $\beta_1$?
		А по $\beta_2$?
		\item Найдите $\hat{\beta}_1$ и $\hat{\beta}_2$ методом максимального правдоподобия. 
		\item Симулируйте $300$ наблюдений $(x_i, y_i)$ таким образом, что $x_i \sim \mathcal{N}(0, 1)$, $y_i = 3e^{x_i}u_i$, $\ln u_i \sim \mathcal{N}(0, 1)$.
		\item По полученным данным найдите $\hat{\beta}_1$ и $\hat{\beta}_2$ в числах.
		\item Проверьте гипотезу
		\[
		H_0: \begin{pmatrix}
			\beta_1 \\
			\beta_2 \\
		\end{pmatrix} = 
		\begin{pmatrix}
			1 \\
			2 \\
		\end{pmatrix}
		\]
		против
		\[
		 H_A: \begin{pmatrix}
			\beta_1 \\
			\beta_2 \\
		\end{pmatrix} \ne
		\begin{pmatrix}
			1 \\
			2 \\
		\end{pmatrix}
		\]
		на уровне значимости $5\%$ при помощи теста $LR$.
	\end{enumerate}
	
\end{document}