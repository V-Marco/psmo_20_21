\documentclass[10pt, a4paper]{extarticle}

%% Язык
\usepackage{cmap} % Поиск в PDF
\usepackage{mathtext} % Кириллица в формулах
\usepackage[T2A]{fontenc} % Кодировка
\usepackage[utf8]{inputenc} % Кодировка
\usepackage[english,russian]{babel} % Локализация, переносы

%% Шрифты

% Serif
%\usepackage{euscript} % Шрифт Евклид
%\usepackage{mathrsfs} % Шрифт для математики
\usepackage{libertinus}

% Sans-serif
%\renewcommand{\rmdefault}{cmss}
%\renewcommand{\ttdefault}{cmss}
%\usepackage{sfmath}

% Настройки для xelatex
%\usepackage{polyglossia} % Для выбора языка в xelatex
%\setmainlanguage{russian}
%\setotherlanguages{english}
% Ligatures=TeX is on by default
% https://tex.stackexchange.com/questions/323542/
%\setmainfont[Ligatures=TeX]{Cantarell}
%\newfontfamily{\cyrillicfonttt}{Times New Roman}
%\newfontfamily\cyrillicfont{Cantarell}[Script=Cyrillic]
%\setsansfont[Ligatures=TeX]{Cantarell}
%\newfontfamily\cyrillicfontsf{Cantarell}[Script=Cyrillic]
%\setmonofont{Courier New}
%\newfontfamily\cyrillicfonttt{Courier New}[Script=Cyrillic]

%% Математика
\usepackage{amsmath, amsfonts, amssymb, amsthm, mathtools}
\usepackage{icomma}

% Операторы
\DeclareMathOperator*\plim{plim}
\DeclareMathOperator{\sgn}{sign}
\DeclareMathOperator{\sign}{sign}
\DeclareMathOperator*{\argmin}{arg\,min}
\DeclareMathOperator*{\argmax}{arg\,max}
\DeclareMathOperator*{\amn}{arg\,min}
\DeclareMathOperator*{\amx}{arg\,max}
\DeclareMathOperator{\cov}{Cov}
\DeclareMathOperator{\Var}{Var}
\DeclareMathOperator{\Cov}{Cov}
\DeclareMathOperator{\Corr}{Corr}
\DeclareMathOperator{\pCorr}{pCorr}
\DeclareMathOperator{\E}{\mathbb{E}}
\let\P\relax
\DeclareMathOperator{\P}{\mathbb{P}}
\renewcommand{\le}{\leqslant}
\renewcommand{\ge}{\geqslant}
\renewcommand{\leq}{\leqslant}
\renewcommand{\geq}{\geqslant}

% Распределения
\newcommand{\cN}{\mathcal{N}}
\newcommand{\cU}{\mathcal{U}}
\newcommand{\cBinom}{\mathcal{Binom}}
\newcommand{\cPois}{\mathcal{Pois}}
\newcommand{\cBeta}{\mathcal{Beta}}
\newcommand{\cGamma}{\mathcal{Gamma}}

% Множества
\def \R{\mathbb{R}}
\def \N{\mathbb{N}}
\def \Z{\mathbb{Z}}

% Другое
\newcommand{\dx}[1]{\,\mathrm{d}#1} % Для интеграла: маленький отступ и прямая d
\newcommand{\ind}[1]{\mathbbm{1}_{\{#1\}}} % Индикатор события
\newcommand{\iid}{\mathrel{\stackrel{\rm i.\,i.\,d.}\sim}}
\newcommand{\const}{\mathrm{const}}

%% Изображения
\usepackage{graphicx}
\usepackage{caption}
\usepackage{subcaption}
\usepackage{physics}
\usepackage{wrapfig} % Обтекание рисунков и таблиц текстом
\usepackage{tikz}

%% Таблицы
\usepackage{array, tabularx, tabulary, booktabs}
\usepackage{longtable}  % Длинные таблицы
\usepackage{multirow} % Слияние строк в таблице

%% Cписки
\usepackage{multicol}
\usepackage{enumitem}

%% Гиперссылки
\usepackage{xcolor}
\usepackage{hyperref}
\definecolor{linkcolor}{HTML}{8b00ff}
\hypersetup{colorlinks = true,
			linkcolor = linkcolor,
			urlcolor = linkcolor,
			citecolor = linkcolor}

%% Выравнивание
\setlength{\parskip}{0.5em} % Расстояние между абзацами
\usepackage{geometry} % Поля
\geometry{
	a4paper,
	left=20mm,
	top=20mm,
	right=20mm}
\setlength{\parindent}{0cm} % Отступ (красная строка)
\linespread{1.0} % Интерлиньяж
\usepackage[many]{tcolorbox}  

%% Оформление

\newtcolorbox{rulesbox}[1]{%
	tikznode boxed title,
	enhanced,
	arc=0mm,
	interior style={white},
	attach boxed title to top center= {yshift=-\tcboxedtitleheight/2},
	fonttitle=\bfseries,
	colbacktitle=white,coltitle=black,
	boxed title style={size=normal,colframe=white,boxrule=0pt},
	title={#1}}

% Красивый серый фон
\usepackage{framed} 
\definecolor{shadecolor}{gray}{0.9}

% Код
\newcommand{\code}[1]{{\tt #1}}

% Колонтитулы
\usepackage{fancyhdr}
\pagestyle{fancy}
\fancyhf{}
\fancyhead[L]{}
\fancyhead[R]{\thepage}

% Разделы и подразделы
\usepackage[sf, sl, outermarks]{titlesec}
\titleformat{\section}{\Large\bfseries\sffamily}{\thesection}{0.5em}{}
\titleformat{\subsection}{\large\sffamily}{\thesubsection}{0.5em}{}

% Содержание
%\usepackage{tocloft}
%\renewcommand{\cftsecfont}{\hspace{4.5em}\normalfont}
%\renewcommand{\cftsubsecfont}{\hspace{5em}\normalfont}
%\renewcommand{\cftsecpagefont}{\normalfont\hfill}
%\renewcommand{\cfttoctitlefont}{\large\normalfont\hfill}
%\renewcommand{\cftaftertoctitle}{\hfill}
%\renewcommand{\cftsecleader}{\cftdotfill{\cftdotsep}}
%\renewcommand{\cftsecafterpnum}{\hspace*{5.5em}\hfill}
%\renewcommand{\cftsubsecafterpnum}{\hspace*{5.5em}\hfill}
%\renewcommand{\cftsecaftersnum}{.}
%\renewcommand{\cftsubsecaftersnum}{.}

%% Комментарии
\usepackage{comment}

%% To-do
\usepackage{todonotes}

%% Литература
\usepackage[backend = biber,
			bibencoding = utf8, 
			sorting = nty, 
			maxcitenames = 4,
			style = numeric-verb]{biblatex}
\addbibresource{lit.bib}
\usepackage{csquotes}

%% Заголовок
\title{{\normalsize Прикладная статистика в машинном обучении} \\\vspace{0.5em} Домашнее задание \#3}
\author{Дедлайн: 17 декабря, 23:59 МСК}
\date{\rule{15cm}{0.4pt}}




\begin{document}
	
	\maketitle
	
	\begin{rulesbox}{Правила игры}
		\begin{enumerate}
			\item Домашнее задание оценивается в $10$ баллов.
			\item Решения принимаются до \textbf{17 декабря 2020 года, 23:59 МСК} включительно.
			Работы, отправленные после дедлайна, не оцениваются.
			\item Все решения нужно загрузить в личный репозиторий на \href{https://classroom.github.com/a/Eh3mAJ78}{GitHub Classroom}.
			\item Обе задачи экспериментальные, и их следует решать в Jupyter Notebook. Однако если вам кажется, что где-то нужно привести длинную теоретическую выкладку, то вы можете написать её от руки и отсканировать. В этом случае соберите все написанные вручную элементы в единый .pdf файл и также загрузите его в репозиторий. 
			\item Репозиторий должен содержать .ipynb-файл с решениями задач, который должен называться name\_surname\_hw3.ipynb. Если вы дополнительно загружаете .pdf файл с решениями от руки, то назовите его add.pdf
			 \item Весь код должен быть написан на Python.
			 \item Разрешается использовать без доказательства любые результаты, встречавшиеся на лекциях или семинарах по курсу, если получение этих результатов не является вопросом задания. 
			 \item Разрешается использовать любые свободные источники с указанием ссылки на них.
			 \item Плагиат не допускается.
			 При обнаружении случаев списывания $0$ за работу выставляется всем участникам нарушения, даже если можно установить, кто у кого списал. 
		\end{enumerate}
	\end{rulesbox}
	
	\newpage

	{\Large \textbf{Задача 1.} Всё нормально! (5 баллов)}
	
	Майло Тэтч исследует выборку независимых одинаково распределённых случайных величин $X_1$, $\ldots$, $X_n$ из нормального $\mathcal{N}(\mu, 1)$ распределения.
	\begin{enumerate}[label = \alph*)]
		\item Симулируйте такую выборку при $n = 100$, $\mu = 5$.
		\item Научный опыт Майло говорит, что $f(\mu)$ непременно равна $1$\footnote{Кажется, что это весьма странное утверждение. Что именно оно означает, мы обсудим на последнем семинаре.}. Найдите апостериорную плотность $f(\mu | X)$ аналитически и постройте её график.
		\item Майло не хочет хранить апостериорную функцию плотности в явном виде и желает представить апостериорное распределение в виде массива чисел. Выберите любой способ сделать это и получите такой массив (размер массива -- на ваш вкус). Постройте гистограмму полученного массива и визуально оцените, насколько хорошо она приближает аналитическую апостериорную плотность.
		\item Майло считает, что недостаточно рассматривать только $\mu$: требуется изучить и $\theta = e^{\mu}$. Найдите апостериорную плотность $f(\theta | X)$ аналитически и при помощи симуляций. Для симуляций используйте следующий алгоритм:
		\begin{enumerate}
			\item Симулируйте $Q_1$, $\ldots$, $Q_B$ из апостериорного распределения $\mu | X$.
			\item Рассчитайте $\gamma_i := e^{Q_i}$. Объявите $\gamma_1$, $\ldots$, $\gamma_B$ выборкой независимых одинаково распределённых случайных величин из апостериорного распределения $\theta | X$.
		\end{enumerate}
		\item Постройте 95\%-ый байесовский доверительный интервал для $\mu$.
		\item Постройте 95\%-ый частотный доверительный интервал для $\mu$. 
	\end{enumerate}

	\begin{flushright}
		\textit{(По мотивам: Wasserman, All of Statistics)}
	\end{flushright}
	\vspace{1em}
	
	{\Large \textbf{Задача 2.} Необычная флора (5 баллов)}
	
	Оказавшись в Атлантиде, Майло Тэтч изучает местную флору. В частности, ему интересно, как влияют привычные удобрения на рост местного вида папоротников. Майло собирает случайную выборку из 500 папоротников и разделяет её на пять непересекающихся групп по 100 наблюдений в каждой. К четырём из них он применяет удобрения разных видов, а пятая группа объявляется контрольной. Результаты эксперимента приведены в таблице ниже. В первой строке указаны виды применяемых удобрений, а во второй -- число растений, для которых наблюдались улучшения в росте. 
	
	\begin{center}
	\begin{tabular}{lrrrrr}
		Вид удобрения      & Без удобрения & DAP & MAP & NPK & UAN \\
		\hline
		Улучшения в росте  & 42		   & 37  & 17  & 74  & 44 \\
	\end{tabular}
	\end{center}

	\begin{enumerate}[label = \alph*)]
		\item Сформулируйте гипотезы о наличии (отсутствии) эффекта действия каждого удобрения. Проверьте каждую гипотезу на уровне значимости 5\%. Можно считать, что число наблюдений достаточно для применения асимптотических результатов. Для получения оценок используйте метод максимального правдоподобия.
		\item Проведите процедуру множественного тестирования методом Бонферрони.
		Изменились ли результаты? 
		
		Напоминание: при полученной статистике $z_{obs}$ p-value можно рассчитать как $2\P\{Z \le z_{obs} | H_0\}$ для двустороннего теста в случае симметричного распределения.
		
		\item Проведите процедуру множественного тестирования методом Бенджамини-Хохберга. Изменились ли результаты? 
		\item Сделайте итоговый вывод о наличии эффекта действия каждого удобрения.
		
		\item Результаты тестирования удобрения MAP смутили Майло: они никак не согласовывались с его научным опытом. Поэтому для исследования эффекта действия этого удобрения Майло решил использовать байесовский подход. Пусть теперь $p_{MAP}$ -- случайная величина, и Майло уверен, что $p_{MAP} \sim \mathrm{Beta}(\alpha, \beta)$. Найдите какую-нибудь (на ваш выбор) точечную байесовскую оценку параметра $p_{MAP}$. Прокомментируйте, насколько сильно различаются частотная и байесовская оценки и поясните почему.
	\end{enumerate}
	
\end{document}