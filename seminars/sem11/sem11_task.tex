\documentclass[10pt, a4paper]{extarticle}

%% Язык
\usepackage{cmap} % Поиск в PDF
\usepackage{mathtext} % Кириллица в формулах
\usepackage[T2A]{fontenc} % Кодировка
\usepackage[utf8]{inputenc} % Кодировка
\usepackage[english,russian]{babel} % Локализация, переносы

%% Шрифты

% Serif
%\usepackage{euscript} % Шрифт Евклид
%\usepackage{mathrsfs} % Шрифт для математики
\usepackage{libertinus}

% Sans-serif
%\renewcommand{\rmdefault}{cmss}
%\renewcommand{\ttdefault}{cmss}
%\usepackage{sfmath}

% Настройки для xelatex
%\usepackage{polyglossia} % Для выбора языка в xelatex
%\setmainlanguage{russian}
%\setotherlanguages{english}
% Ligatures=TeX is on by default
% https://tex.stackexchange.com/questions/323542/
%\setmainfont[Ligatures=TeX]{Cantarell}
%\newfontfamily{\cyrillicfonttt}{Times New Roman}
%\newfontfamily\cyrillicfont{Cantarell}[Script=Cyrillic]
%\setsansfont[Ligatures=TeX]{Cantarell}
%\newfontfamily\cyrillicfontsf{Cantarell}[Script=Cyrillic]
%\setmonofont{Courier New}
%\newfontfamily\cyrillicfonttt{Courier New}[Script=Cyrillic]

%% Математика
\usepackage{amsmath, amsfonts, amssymb, amsthm, mathtools}
\usepackage{icomma}

\newtheorem{theor}{Теорема}
\newtheorem{defn}{Определение}

% Операторы
\DeclareMathOperator*\plim{plim}
\DeclareMathOperator{\sgn}{sign}
\DeclareMathOperator{\sign}{sign}
\DeclareMathOperator*{\argmin}{arg\,min}
\DeclareMathOperator*{\argmax}{arg\,max}
\DeclareMathOperator*{\amn}{arg\,min}
\DeclareMathOperator*{\amx}{arg\,max}
\DeclareMathOperator{\cov}{Cov}
\DeclareMathOperator{\Var}{Var}
\DeclareMathOperator{\Cov}{Cov}
\DeclareMathOperator{\Corr}{Corr}
\DeclareMathOperator{\pCorr}{pCorr}
\DeclareMathOperator{\E}{\mathbb{E}}
\let\P\relax
\DeclareMathOperator{\P}{\mathbb{P}}
\renewcommand{\le}{\leqslant}
\renewcommand{\ge}{\geqslant}
\renewcommand{\leq}{\leqslant}
\renewcommand{\geq}{\geqslant}

% Распределения
\newcommand{\cN}{\mathcal{N}}
\newcommand{\cU}{\mathcal{U}}
\newcommand{\cBinom}{\mathcal{Binom}}
\newcommand{\cPois}{\mathcal{Pois}}
\newcommand{\cBeta}{\mathcal{Beta}}
\newcommand{\cGamma}{\mathcal{Gamma}}

% Множества
\def \R{\mathbb{R}}
\def \N{\mathbb{N}}
\def \Z{\mathbb{Z}}

% Другое
\newcommand{\dx}[1]{\,\mathrm{d}#1} % Для интеграла: маленький отступ и прямая d
\newcommand{\ind}[1]{\mathbbm{1}_{\{#1\}}} % Индикатор события
\newcommand{\iid}{\mathrel{\stackrel{\rm i.\,i.\,d.}\sim}}
\newcommand{\const}{\mathrm{const}}

%% Изображения
\usepackage{graphicx}
\usepackage{caption}
\usepackage{subcaption}
\usepackage{physics}
\usepackage{wrapfig} % Обтекание рисунков и таблиц текстом
\usepackage{tikz}

%% Таблицы
\usepackage{array, tabularx, tabulary, booktabs}
\usepackage{longtable}  % Длинные таблицы
\usepackage{multirow} % Слияние строк в таблице

%% Cписки
\usepackage{multicol}
\usepackage{enumitem}

%% Гиперссылки
\usepackage{xcolor}
\usepackage{hyperref}
\definecolor{linkcolor}{HTML}{8b00ff}
\hypersetup{colorlinks = true,
			linkcolor = linkcolor,
			urlcolor = linkcolor,
			citecolor = linkcolor}

%% Выравнивание
\setlength{\parskip}{0.5em} % Расстояние между абзацами
\usepackage{geometry} % Поля
\geometry{
	a4paper,
	left=20mm,
	top=20mm,
	right=20mm}
\setlength{\parindent}{0cm} % Отступ (красная строка)
\linespread{1.0} % Интерлиньяж
\usepackage[many]{tcolorbox}  

%% Оформление

\newtcolorbox{rulesbox}[1]{%
	tikznode boxed title,
	enhanced,
	arc=0mm,
	interior style={white},
	attach boxed title to top center= {yshift=-\tcboxedtitleheight/2},
	fonttitle=\bfseries,
	colbacktitle=white,coltitle=black,
	boxed title style={size=normal,colframe=white,boxrule=0pt},
	title={#1}}

% Красивый серый фон
\usepackage{framed} 
\definecolor{shadecolor}{gray}{0.9}

% Код
\newcommand{\code}[1]{{\tt #1}}

% Колонтитулы
\usepackage{fancyhdr}
\pagestyle{fancy}
\fancyhf{}
\fancyhead[L]{}
\fancyhead[R]{\thepage}

% Разделы и подразделы
\usepackage[sf, sl, outermarks]{titlesec}
\titleformat{\section}{\Large\bfseries\sffamily}{\thesection}{0.5em}{}
\titleformat{\subsection}{\large\sffamily}{\thesubsection}{0.5em}{}

% Содержание
%\usepackage{tocloft}
%\renewcommand{\cftsecfont}{\hspace{4.5em}\normalfont}
%\renewcommand{\cftsubsecfont}{\hspace{5em}\normalfont}
%\renewcommand{\cftsecpagefont}{\normalfont\hfill}
%\renewcommand{\cfttoctitlefont}{\large\normalfont\hfill}
%\renewcommand{\cftaftertoctitle}{\hfill}
%\renewcommand{\cftsecleader}{\cftdotfill{\cftdotsep}}
%\renewcommand{\cftsecafterpnum}{\hspace*{5.5em}\hfill}
%\renewcommand{\cftsubsecafterpnum}{\hspace*{5.5em}\hfill}
%\renewcommand{\cftsecaftersnum}{.}
%\renewcommand{\cftsubsecaftersnum}{.}

%% Комментарии
\usepackage{comment}

%% To-do
\usepackage{todonotes}

%% Литература
\usepackage[backend = biber,
			bibencoding = utf8, 
			sorting = nty, 
			maxcitenames = 4,
			style = numeric-verb]{biblatex}
\addbibresource{lit.bib}
\usepackage{csquotes}

%% Заголовок
\title{{\normalsize Прикладная статистика в машинном обучении} \\ \vspace{0.5em} Семинар 11}
\author{\rule{15cm}{0.4pt}}




\begin{document}
	
	\maketitle

	{\Large \textbf{Задача 1.} Теорема Гаусса-Маркова для стохастических регрессоров.}
	\vspace{1em}
	
	Если:
	\begin{enumerate}
		\item Модель задана как $y = X\beta + u$, и при помощи МНК оценивается линейная регрессия $\hat{y} = X\hat{\beta}$.
		\item $\beta$ -- вектор констант.
		\item $\P(\{\text{у матрицы $X$ есть линейно зависимые столбцы}\}) = 0$.
		\item $\E(u | X) = 0$, $\Var(u | X) = \sigma^2 I$,
	\end{enumerate}
	то
	\begin{enumerate}
		\item $\hat{\beta}$ существуют и единственны с вероятностью один.
		\item $\hat{\beta}$ линейны по $y$.
		\item $\E(\hat{\beta} | X) = \beta$.
		\item $\hat{\beta}$ эффективны в классе линейных по $y$ и условно несмещённых оценок. 
	\end{enumerate}
	
	\begin{enumerate}[label = \alph*)]
		\item Покажите, что если предпосылки ТГМ со стохастическими регрессорами выполнены, то оценки $\hat{\beta}$ являются состоятельными.
		\item При нарушении какого условия ТГМ оценки МНК перестанут быть состоятельными? 
	\end{enumerate}
	\vspace{1em}

	{\Large \textbf{Задача 2.} Причины эндогенности.}
	
	\begin{enumerate}[label=\alph*)]
		\item Рассмотрим модель $y_i = \beta x_i + u_i$ со стохастическими регрессорами, в которой выполнены все предпосылки ТГМ, кроме $\E(u_i | X) \ne 0$ по какой-то причине. Покажите, что если оценивается регрессия $\hat{y}_i = \hat{\beta}x_i$ при помощи МНК, то $\hat{\beta}$ не являются состоятельными.
		\item \textit{(Пропущенные переменные)} Рассмотрим модель $y_i = \beta x_i + \gamma z_i + u_i$ со стохастическими переменными, в которой выполнены все предпосылки ТГМ. Исследователь Дорофей оценивает регрессию $\hat{y}_i = \hat{\beta}x_i$ при помощи МНК. Покажите, что оценки в регрессии Дорофея не являются состоятельными.
		\item \textit{(Ошибки измерения)} Отчаянный исследователь Афанасий пытается предсказать результаты контрольной по статистике при помощи модели со стохастическими регрессорами $y_i = \beta x_i + u_i$, где $x_i$ -- количество съеденных бургеров известной фирмы. Для сбора данных производится опрос студентов. Афанасий догадывается, что полученные данные отражают не настоящие $x_i$, а строятся следующим образом:
		\[
		x_i^* = x_i + \alpha + v_i,
		\]
		где $\alpha$ -- показатель русской народной скромности, а $v_i$ -- случайная величина, отражающая несовершенство человеческой памяти. Предположим, что $\alpha$ -- константа, а $\E(x_i) = \mu_x$, $\E(u_i) = 0$, $\E(v_i) = 0$, $\Var(x_i) = \sigma^2_x$, $\Var(u_i) = \sigma^2_u$, $\Var(v_i) = \sigma^2_v$. Предположим, что $x_i$ не коррелирует с $v_i$ и $u_i$, а все $u_i$ и $v_i$ не коррелируют друг с другом и между собой.
		
		Афанасий оценивает регрессию $\hat{y}_i = \hat{\beta} x^*_i$ при помощи МНК по собранным данным. Покажите, что $\hat{\beta}$ не является состоятельной. Как смещение зависит от $\alpha$?
		
	\end{enumerate}
	\vspace{1em}
	
	{\Large \textbf{Задача 3.} Двухшаговый МНК.}
	
	Переменные $z_i$ называются инструментальными, если 
	\begin{enumerate}
		\item \textit{(релевантность)} $\E(x_iz_i) \ne 0 \sim \cov(x_i, z_i) \ne 0 \sim \plim_{n \to \infty} Z'_jX \ne 0$.
		\item \textit{(валидность)} $\E(u_iz_i) = 0 \sim \cov(u_i, z_i) = 0 \sim \plim_{n \to \infty} Z'_ju = 0$.
	\end{enumerate}

	Если подобрали подходящие инструменты, то можно применить двухшаговый МНК:
	\begin{description}
		\item[Шаг 1.] Оценить регрессию $X$ на $Z$, получить прогнозы $\hat{X}$.
		\item[Шаг 2.] Оценить регрессию $y$ на $\hat{X}$, получить оценки $\hat{\beta}_{2SLS}$.
	\end{description}

	\begin{enumerate}[label=\alph*)]
		\item Выведите оценки двухшагового МНК в общем случае ($Z$ имеет размеры $n\times m$, $m \ge k$).
		\item Выведите оценки метода инструментальных переменных ($Z$ имеет размеры $n\times m$, $m = k$).
		\item Покажите, что оценки метода инструментальных переменных являются состоятельными.
	\end{enumerate}
	
\end{document}