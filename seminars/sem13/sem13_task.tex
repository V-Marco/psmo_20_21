\documentclass[10pt, a4paper]{extarticle}

%% Язык
\usepackage{cmap} % Поиск в PDF
\usepackage{mathtext} % Кириллица в формулах
\usepackage[T2A]{fontenc} % Кодировка
\usepackage[utf8]{inputenc} % Кодировка
\usepackage[english,russian]{babel} % Локализация, переносы

%% Шрифты

% Serif
%\usepackage{euscript} % Шрифт Евклид
%\usepackage{mathrsfs} % Шрифт для математики
\usepackage{libertinus}

% Sans-serif
%\renewcommand{\rmdefault}{cmss}
%\renewcommand{\ttdefault}{cmss}
%\usepackage{sfmath}

% Настройки для xelatex
%\usepackage{polyglossia} % Для выбора языка в xelatex
%\setmainlanguage{russian}
%\setotherlanguages{english}
% Ligatures=TeX is on by default
% https://tex.stackexchange.com/questions/323542/
%\setmainfont[Ligatures=TeX]{Cantarell}
%\newfontfamily{\cyrillicfonttt}{Times New Roman}
%\newfontfamily\cyrillicfont{Cantarell}[Script=Cyrillic]
%\setsansfont[Ligatures=TeX]{Cantarell}
%\newfontfamily\cyrillicfontsf{Cantarell}[Script=Cyrillic]
%\setmonofont{Courier New}
%\newfontfamily\cyrillicfonttt{Courier New}[Script=Cyrillic]

%% Математика
\usepackage{amsmath, amsfonts, amssymb, amsthm, mathtools}
\usepackage{icomma}

\newtheorem{theor}{Теорема}
\newtheorem{defn}{Определение}

% Операторы
\DeclareMathOperator*\plim{plim}
\DeclareMathOperator{\sgn}{sign}
\DeclareMathOperator{\sign}{sign}
\DeclareMathOperator*{\argmin}{arg\,min}
\DeclareMathOperator*{\argmax}{arg\,max}
\DeclareMathOperator*{\amn}{arg\,min}
\DeclareMathOperator*{\amx}{arg\,max}
\DeclareMathOperator{\cov}{Cov}
\DeclareMathOperator{\Var}{Var}
\DeclareMathOperator{\Cov}{Cov}
\DeclareMathOperator{\Corr}{Corr}
\DeclareMathOperator{\pCorr}{pCorr}
\DeclareMathOperator{\E}{\mathbb{E}}
\let\P\relax
\DeclareMathOperator{\P}{\mathbb{P}}
\renewcommand{\le}{\leqslant}
\renewcommand{\ge}{\geqslant}
\renewcommand{\leq}{\leqslant}
\renewcommand{\geq}{\geqslant}

% Распределения
\newcommand{\cN}{\mathcal{N}}
\newcommand{\cU}{\mathcal{U}}
\newcommand{\cBinom}{\mathcal{Binom}}
\newcommand{\cPois}{\mathcal{Pois}}
\newcommand{\cBeta}{\mathcal{Beta}}
\newcommand{\cGamma}{\mathcal{Gamma}}

% Множества
\def \R{\mathbb{R}}
\def \N{\mathbb{N}}
\def \Z{\mathbb{Z}}

% Другое
\newcommand{\dx}[1]{\,\mathrm{d}#1} % Для интеграла: маленький отступ и прямая d
\newcommand{\ind}[1]{\mathbbm{1}_{\{#1\}}} % Индикатор события
\newcommand{\iid}{\mathrel{\stackrel{\rm i.\,i.\,d.}\sim}}
\newcommand{\const}{\mathrm{const}}

%% Изображения
\usepackage{graphicx}
\usepackage{caption}
\usepackage{subcaption}
\usepackage{physics}
\usepackage{wrapfig} % Обтекание рисунков и таблиц текстом
\usepackage{tikz}

%% Таблицы
\usepackage{array, tabularx, tabulary, booktabs}
\usepackage{longtable}  % Длинные таблицы
\usepackage{multirow} % Слияние строк в таблице

%% Cписки
\usepackage{multicol}
\usepackage{enumitem}

%% Гиперссылки
\usepackage{xcolor}
\usepackage{hyperref}
\definecolor{linkcolor}{HTML}{8b00ff}
\hypersetup{colorlinks = true,
			linkcolor = linkcolor,
			urlcolor = linkcolor,
			citecolor = linkcolor}

%% Выравнивание
\setlength{\parskip}{0.5em} % Расстояние между абзацами
\usepackage{geometry} % Поля
\geometry{
	a4paper,
	left=20mm,
	top=20mm,
	right=20mm}
\setlength{\parindent}{0cm} % Отступ (красная строка)
\linespread{1.0} % Интерлиньяж
\usepackage[many]{tcolorbox}  

%% Оформление

\newtcolorbox{rulesbox}[1]{%
	tikznode boxed title,
	enhanced,
	arc=0mm,
	interior style={white},
	attach boxed title to top center= {yshift=-\tcboxedtitleheight/2},
	fonttitle=\bfseries,
	colbacktitle=white,coltitle=black,
	boxed title style={size=normal,colframe=white,boxrule=0pt},
	title={#1}}

% Красивый серый фон
\usepackage{framed} 
\definecolor{shadecolor}{gray}{0.9}

% Код
\newcommand{\code}[1]{{\tt #1}}

% Колонтитулы
\usepackage{fancyhdr}
\pagestyle{fancy}
\fancyhf{}
\fancyhead[L]{}
\fancyhead[R]{\thepage}

% Разделы и подразделы
\usepackage[sf, sl, outermarks]{titlesec}
\titleformat{\section}{\Large\bfseries\sffamily}{\thesection}{0.5em}{}
\titleformat{\subsection}{\large\sffamily}{\thesubsection}{0.5em}{}

% Содержание
%\usepackage{tocloft}
%\renewcommand{\cftsecfont}{\hspace{4.5em}\normalfont}
%\renewcommand{\cftsubsecfont}{\hspace{5em}\normalfont}
%\renewcommand{\cftsecpagefont}{\normalfont\hfill}
%\renewcommand{\cfttoctitlefont}{\large\normalfont\hfill}
%\renewcommand{\cftaftertoctitle}{\hfill}
%\renewcommand{\cftsecleader}{\cftdotfill{\cftdotsep}}
%\renewcommand{\cftsecafterpnum}{\hspace*{5.5em}\hfill}
%\renewcommand{\cftsubsecafterpnum}{\hspace*{5.5em}\hfill}
%\renewcommand{\cftsecaftersnum}{.}
%\renewcommand{\cftsubsecaftersnum}{.}

%% Комментарии
\usepackage{comment}

%% To-do
\usepackage{todonotes}

%% Литература
\usepackage[backend = biber,
			bibencoding = utf8, 
			sorting = nty, 
			maxcitenames = 4,
			style = numeric-verb]{biblatex}
\addbibresource{lit.bib}
\usepackage{csquotes}

%% Заголовок
\title{{\normalsize Прикладная статистика в машинном обучении} \\ \vspace{0.5em} Семинар 13}
\author{\rule{15cm}{0.4pt}}




\begin{document}
	
	\maketitle

	{\Large \textbf{Задача 1.}} 
	
	Пусть $X_1$, $\ldots$, $X_n$ -- выборка независимых одинаково распределённых случайных величин, имеющих распределение Бернулли с параметром $p$. Рассмотрим в качестве априорного распределения параметра $p$ равномерное на подходящем интервале, то есть $f(p) = 1$.
	\begin{enumerate}[label = \alph*)]
		\item Выпишите апостериорное распределение $p$ с точностью до константы.
		\item Бета-распределение с параметрами $\alpha$ и $\beta$ задаётся плотностью:
		\[
		f(x | \alpha, \beta) = \dfrac{\Gamma(\alpha + \beta)}{\Gamma(\alpha)\Gamma(\beta)}x^{\alpha - 1}(1-x)^{\beta - 1}.
		\]
		Подберите такие параметры $\alpha$ и $\beta$, чтобы показать, что полученное апостериорное распределение представимо в виде $p|x \sim C\mathrm{Beta}(\alpha, \beta)$, где $C$ -- некоторая константа.
		\item Восстановите константу $C$.
		\item Найдите точечную оценку параметра $p$.
		\item Заметим, что $\bar{p}$ представляется в виде
		\[
		\bar{p} = \dfrac{n}{n + 2}\dfrac{s}{n} + \left(1 - \dfrac{n}{n+2}\right)\dfrac{1}{2}.
		\]
		Прокомментируйте интуицию такого разложения.
		\item Запишите формулу расчёта 95\%-го апостериорного доверительного интервала для $p$.
		\item Теперь предположим, что априорное распределение параметра $p$ -- это Бета-распределение $p \sim \mathrm{Beta}(\alpha, \beta)$. Вычислите апостериорное распределение $p$.
	\end{enumerate}
	\vspace{1em}
	
	{\Large \textbf{Задача 2.}} 
	
	Рассмотрим выборку независимых одинаково распределённых нормальных случайных величин $X_1$, $\ldots$, $X_n$ $\sim \mathcal{N}(\mu, \sigma^2)$. Для простоты будем считать, что $\sigma$ известна. Пусть априорное распределение параметра $\mu$ имеет вид $\mu \sim \mathcal{N}(a, b^2)$.
	
	\begin{enumerate}[label = \alph*)]
		\item Пусть функция правдоподобия строится лишь по наблюдению $X_1$. Выведите примерное апостериорное распределение параметра $\mu$. Является ли априорное распределение сопряжённым?
		\item Можно показать, что в общем случае апостериорное распределение $\mu$ имеет вид
		\[
		\mu | X \sim \mathcal{N}(\bar{\mu}, w^2),
		\]
		где $\bar{\mu} = \lambda \bar{X} + (1 - \lambda)a$, $\lambda = \dfrac{1/s^2}{1/s^2 + 1/b^2}$, $s = \sigma / \sqrt{n}$, $1/w^2 = 1/s^2 + 1/b^2$.
		
		Что происходит с $\lambda$ и $w/s$ при $n \to \infty$? Выпишите приблизительное апостериорное распределение $\mu$ при $n \to \infty$.
		\item Постройте 95\% асимптотический байесовский доверительный интервал для $\mu$.
	\end{enumerate}
	\vspace{1em}


	{\Large \textbf{Задача 3.}}
	
	Пусть $X_1$, $\ldots$, $X_n$ -- выборка независимых одинаково распределённых случайных величин из распределения Бернулли с параметром $p$. Апостериорное распределение задаётся как $f(p) = 1$. Пусть $\psi = \log\left(\dfrac{p}{1-p}\right)$. Найдите апостериорную функцию распределения и функцию плотности $\psi$.
	\vspace{1em}
	  
	
	{\Large \textbf{Задача 4.}}
	
	Компания тестирует новый вид удобрений, для чего создаёт контрольную группу размером $n_1$ и тестовую группу размером $n_2$. После проведения эксперимента оказалось, что в контрольной группе лучший рост показали $X_1$ растений, а в тестовой -- $X_2$ растений. Пусть $f(p_1, p_2) = 1$. 
	
	\begin{enumerate}[label = \alph*)]
		\item Выведите апостериорное распределение $f(p_1, p_2 | x_1, x_2)$ с точностью до константы.
		\item Являются ли $p_1 | x_1$ и $p_2 | x_2$ независимыми?
		\item Опишите процесс байесовской оценки величины $p_2 - p_1$ через симуляцию.
	\end{enumerate}

\end{document}