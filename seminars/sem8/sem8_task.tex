\documentclass[10pt, a4paper]{extarticle}

%% Язык
\usepackage{cmap} % Поиск в PDF
\usepackage{mathtext} % Кириллица в формулах
\usepackage[T2A]{fontenc} % Кодировка
\usepackage[utf8]{inputenc} % Кодировка
\usepackage[english,russian]{babel} % Локализация, переносы

%% Шрифты

% Serif
%\usepackage{euscript} % Шрифт Евклид
%\usepackage{mathrsfs} % Шрифт для математики
\usepackage{libertinus}

% Sans-serif
%\renewcommand{\rmdefault}{cmss}
%\renewcommand{\ttdefault}{cmss}
%\usepackage{sfmath}

% Настройки для xelatex
%\usepackage{polyglossia} % Для выбора языка в xelatex
%\setmainlanguage{russian}
%\setotherlanguages{english}
% Ligatures=TeX is on by default
% https://tex.stackexchange.com/questions/323542/
%\setmainfont[Ligatures=TeX]{Cantarell}
%\newfontfamily{\cyrillicfonttt}{Times New Roman}
%\newfontfamily\cyrillicfont{Cantarell}[Script=Cyrillic]
%\setsansfont[Ligatures=TeX]{Cantarell}
%\newfontfamily\cyrillicfontsf{Cantarell}[Script=Cyrillic]
%\setmonofont{Courier New}
%\newfontfamily\cyrillicfonttt{Courier New}[Script=Cyrillic]

%% Математика
\usepackage{amsmath, amsfonts, amssymb, amsthm, mathtools}
\usepackage{icomma}

\newtheorem{theor}{Теорема}
\newtheorem{defn}{Определение}

% Операторы
\DeclareMathOperator*\plim{plim}
\DeclareMathOperator{\sgn}{sign}
\DeclareMathOperator{\sign}{sign}
\DeclareMathOperator*{\argmin}{arg\,min}
\DeclareMathOperator*{\argmax}{arg\,max}
\DeclareMathOperator*{\amn}{arg\,min}
\DeclareMathOperator*{\amx}{arg\,max}
\DeclareMathOperator{\cov}{Cov}
\DeclareMathOperator{\Var}{Var}
\DeclareMathOperator{\Cov}{Cov}
\DeclareMathOperator{\Corr}{Corr}
\DeclareMathOperator{\pCorr}{pCorr}
\DeclareMathOperator{\E}{\mathbb{E}}
\let\P\relax
\DeclareMathOperator{\P}{\mathbb{P}}
\renewcommand{\le}{\leqslant}
\renewcommand{\ge}{\geqslant}
\renewcommand{\leq}{\leqslant}
\renewcommand{\geq}{\geqslant}

% Распределения
\newcommand{\cN}{\mathcal{N}}
\newcommand{\cU}{\mathcal{U}}
\newcommand{\cBinom}{\mathcal{Binom}}
\newcommand{\cPois}{\mathcal{Pois}}
\newcommand{\cBeta}{\mathcal{Beta}}
\newcommand{\cGamma}{\mathcal{Gamma}}

% Множества
\def \R{\mathbb{R}}
\def \N{\mathbb{N}}
\def \Z{\mathbb{Z}}

% Другое
\newcommand{\dx}[1]{\,\mathrm{d}#1} % Для интеграла: маленький отступ и прямая d
\newcommand{\ind}[1]{\mathbbm{1}_{\{#1\}}} % Индикатор события
\newcommand{\iid}{\mathrel{\stackrel{\rm i.\,i.\,d.}\sim}}
\newcommand{\const}{\mathrm{const}}

%% Изображения
\usepackage{graphicx}
\usepackage{caption}
\usepackage{subcaption}
\usepackage{physics}
\usepackage{wrapfig} % Обтекание рисунков и таблиц текстом
\usepackage{tikz}

%% Таблицы
\usepackage{array, tabularx, tabulary, booktabs}
\usepackage{longtable}  % Длинные таблицы
\usepackage{multirow} % Слияние строк в таблице

%% Cписки
\usepackage{multicol}
\usepackage{enumitem}

%% Гиперссылки
\usepackage{xcolor}
\usepackage{hyperref}
\definecolor{linkcolor}{HTML}{8b00ff}
\hypersetup{colorlinks = true,
			linkcolor = linkcolor,
			urlcolor = linkcolor,
			citecolor = linkcolor}

%% Выравнивание
\setlength{\parskip}{0.5em} % Расстояние между абзацами
\usepackage{geometry} % Поля
\geometry{
	a4paper,
	left=20mm,
	top=20mm,
	right=20mm}
\setlength{\parindent}{0cm} % Отступ (красная строка)
\linespread{1.0} % Интерлиньяж
\usepackage[many]{tcolorbox}  

%% Оформление

\newtcolorbox{rulesbox}[1]{%
	tikznode boxed title,
	enhanced,
	arc=0mm,
	interior style={white},
	attach boxed title to top center= {yshift=-\tcboxedtitleheight/2},
	fonttitle=\bfseries,
	colbacktitle=white,coltitle=black,
	boxed title style={size=normal,colframe=white,boxrule=0pt},
	title={#1}}

% Красивый серый фон
\usepackage{framed} 
\definecolor{shadecolor}{gray}{0.9}

% Код
\newcommand{\code}[1]{{\tt #1}}

% Колонтитулы
\usepackage{fancyhdr}
\pagestyle{fancy}
\fancyhf{}
\fancyhead[L]{}
\fancyhead[R]{\thepage}

% Разделы и подразделы
\usepackage[sf, sl, outermarks]{titlesec}
\titleformat{\section}{\Large\bfseries\sffamily}{\thesection}{0.5em}{}
\titleformat{\subsection}{\large\sffamily}{\thesubsection}{0.5em}{}

% Содержание
%\usepackage{tocloft}
%\renewcommand{\cftsecfont}{\hspace{4.5em}\normalfont}
%\renewcommand{\cftsubsecfont}{\hspace{5em}\normalfont}
%\renewcommand{\cftsecpagefont}{\normalfont\hfill}
%\renewcommand{\cfttoctitlefont}{\large\normalfont\hfill}
%\renewcommand{\cftaftertoctitle}{\hfill}
%\renewcommand{\cftsecleader}{\cftdotfill{\cftdotsep}}
%\renewcommand{\cftsecafterpnum}{\hspace*{5.5em}\hfill}
%\renewcommand{\cftsubsecafterpnum}{\hspace*{5.5em}\hfill}
%\renewcommand{\cftsecaftersnum}{.}
%\renewcommand{\cftsubsecaftersnum}{.}

%% Комментарии
\usepackage{comment}

%% To-do
\usepackage{todonotes}

%% Литература
\usepackage[backend = biber,
			bibencoding = utf8, 
			sorting = nty, 
			maxcitenames = 4,
			style = numeric-verb]{biblatex}
\addbibresource{lit.bib}
\usepackage{csquotes}

%% Заголовок
\title{{\normalsize Прикладная статистика в машинном обучении} \\ \vspace{0.5em} Семинар 8}
\author{\rule{15cm}{0.4pt}}




\begin{document}
	
	\maketitle

	{\Large \textbf{Задача 1.} Тестирование гипотез в множественной регрессии.}
	
	Линейная регрессионная модель задаётся в следующем виде:
	\[
	y_i = \beta_0 + \beta_1x_{1, i} + \beta_2x_{2, i} + u_i.
	\] 
	Предположим, что $u \sim \mathcal{N}(0, \sigma^2I)$. Известно, что
	\[
	X = \begin{pmatrix}
		1 & 0 & 0 \\
		1 & 0 & 0 \\
		1 & 0 & 0 \\
		1 & 1 & 0 \\
		1 & 1 & 1 \\
	\end{pmatrix},\text{  }
	y = \begin{pmatrix}
		1 \\
		2 \\
		3 \\
		4 \\
		5
	\end{pmatrix}, \text{ }
	(X'X)^{-1} = \begin{pmatrix}
		1/3 & -1/3 & 0 \\
		-1/3 & 4/3 & -1 \\
		0 & -1 & 2
	\end{pmatrix}
	\]
	
	\begin{enumerate}[label = \alph*)]
		\item Найдите $\hat{\beta_0}$, $\hat{\beta_1}$, $\hat{\beta_2}$.
		\item Найдите $\hat{y}$.
		\item Найдите TSS, ESS, RSS и $R^2$.
		\item Найдите $\hat{\sigma}$.
		\item Найдите $\widehat{\Var}(\hat{\beta})$.
		\item Здесь и далее все гипотезы проверяются на уровне значимости 5\%. Проверьте гипотезу $H_0$ против $H_1$, если
		\[
		\begin{cases}
		H_0: \beta_1 = 1, \\
		H_1: \beta_1 \ne 1.
		\end{cases}
		\]
		\item Проверьте гипотезу $H_0$ против $H_1$, если
		\[
		\begin{cases}
			H_0: \beta_1 = 1, \\
			H_1: \beta_1 > 1.
		\end{cases}
		\]
		\item Проверьте регрессию на значимость в целом.
		\item Проверьте гипотезу
		\[
		\begin{cases}
			H_0: \beta_2 = \beta_3, \\
			H_1: \beta_2 \ne \beta_3.
		\end{cases}
		\]
		\item Постройте $95\%$-ый доверительный интервал для $\beta_1$.
		\item Пусть $x_{1, 6} = 10$, $x_{2, 6} = 7$. Найдите $\hat{y}_6$.
		\item Постройте $95\%$-ый доверительный интервал для $\E(y_6 | X_6)$.
	\end{enumerate}
	\newpage
	
	
	{\Large \textbf{Задача 2.} Мегаматрица.}
	
	Правила работы с многомерными случайными величинами ($y$, $z$ -- случайные векторы, $A$, $B$ -- матрицы констант; считаем, что все размеры подходящие):
	\begin{enumerate}
		\item $\E(Ay) = A\E(y)$.
		\item $\Var(Ay) = A\Var(y)A'$.
		\item $\Var(y + z) = \Var(y) + \Var(z) + \cov(y, z) + \cov(z, y)$.
		\item $\cov(Ay, Bz) = A\cov(y, z)B'$.
	\end{enumerate}

	Рассмотрим линейную модель $y = X\beta + u$, оцениваемую при помощи МНК. Пусть $\E(u) = 0$, $\Var(u) = \sigma^2 I$, число наблюдений равно $n$, число регрессоров, включая константный, равно $k$. 

	Заполните матрицу характеристик элементов МНК:
	\begin{center}
	\begin{tabular}{c|ccccc}
		$\Var(\cdot)$ & $y$ & $\hat{y}$ & $\hat{\beta}$ & $\hat{u}$ & $u$ \\
		\hline
		$y$ & $\ldots$ &&&& \\
		$\hat{y}$ &&$\ldots$&&& \\
		$\hat{\beta}$ &&&$\ldots$&& \\
		$\hat{u}$ &&&&$\ldots$& \\
		$u$ &&&&&$\ldots$ \\
		\hline
		$\E(\cdot)$ &&&&& \\
		\hline
	\end{tabular}
\end{center}

	Для каждого элемента укажите размеры.

	\begin{enumerate}[label = \alph*)]
		\item $\E(y)$
		\item $\E(\hat{\beta})$
		\item $\Var(y)$
		\item $\Var(\hat{\beta})$
		\item $\cov(\hat{\beta}, \hat{u})$
		\item $\ldots$
	\end{enumerate}
	
\end{document}