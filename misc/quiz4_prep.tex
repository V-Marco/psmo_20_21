\documentclass[10pt, a4paper]{exam}

%% Язык
\usepackage{cmap} % Поиск в PDF
\usepackage{mathtext} % Кириллица в формулах
\usepackage[T2A]{fontenc} % Кодировка
\usepackage[utf8]{inputenc} % Кодировка
\usepackage[english,russian]{babel} % Локализация, переносы

%% Шрифты

% Serif
%\usepackage{euscript} % Шрифт Евклид
%\usepackage{mathrsfs} % Шрифт для математики
\usepackage{libertinus}

% Sans-serif
%\renewcommand{\rmdefault}{cmss}
%\renewcommand{\ttdefault}{cmss}
%\usepackage{sfmath}

% Настройки для xelatex
%\usepackage{polyglossia} % Для выбора языка в xelatex
%\setmainlanguage{russian}
%\setotherlanguages{english}
% Ligatures=TeX is on by default
% https://tex.stackexchange.com/questions/323542/
%\setmainfont[Ligatures=TeX]{Cantarell}
%\newfontfamily{\cyrillicfonttt}{Times New Roman}
%\newfontfamily\cyrillicfont{Cantarell}[Script=Cyrillic]
%\setsansfont[Ligatures=TeX]{Cantarell}
%\newfontfamily\cyrillicfontsf{Cantarell}[Script=Cyrillic]
%\setmonofont{Courier New}
%\newfontfamily\cyrillicfonttt{Courier New}[Script=Cyrillic]

%% Математика
\usepackage{amsmath, amsfonts, amssymb, amsthm, mathtools}
\usepackage{icomma}

% Операторы
\DeclareMathOperator*\plim{plim}
\DeclareMathOperator{\sgn}{sign}
\DeclareMathOperator{\sign}{sign}
\DeclareMathOperator*{\argmin}{arg\,min}
\DeclareMathOperator*{\argmax}{arg\,max}
\DeclareMathOperator*{\amn}{arg\,min}
\DeclareMathOperator*{\amx}{arg\,max}
\DeclareMathOperator{\cov}{Cov}
\DeclareMathOperator{\Var}{Var}
\DeclareMathOperator{\Cov}{Cov}
\DeclareMathOperator{\Corr}{Corr}
\DeclareMathOperator{\pCorr}{pCorr}
\DeclareMathOperator{\E}{\mathbb{E}}
\let\P\relax
\DeclareMathOperator{\P}{\mathbb{P}}
\renewcommand{\le}{\leqslant}
\renewcommand{\ge}{\geqslant}
\renewcommand{\leq}{\leqslant}
\renewcommand{\geq}{\geqslant}

% Распределения
\newcommand{\cN}{\mathcal{N}}
\newcommand{\cU}{\mathcal{U}}
\newcommand{\cBinom}{\mathcal{Binom}}
\newcommand{\cPois}{\mathcal{Pois}}
\newcommand{\cBeta}{\mathcal{Beta}}
\newcommand{\cGamma}{\mathcal{Gamma}}

% Множества
\def \R{\mathbb{R}}
\def \N{\mathbb{N}}
\def \Z{\mathbb{Z}}

% Другое
\newcommand{\dx}[1]{\,\mathrm{d}#1} % Для интеграла: маленький отступ и прямая d
\newcommand{\ind}[1]{\mathbbm{1}_{\{#1\}}} % Индикатор события
\newcommand{\iid}{\mathrel{\stackrel{\rm i.\,i.\,d.}\sim}}
\newcommand{\const}{\mathrm{const}}

%% Изображения
\usepackage{graphicx}
\usepackage{caption}
\usepackage{subcaption}
\usepackage{physics}
\usepackage{wrapfig} % Обтекание рисунков и таблиц текстом
\usepackage{tikz}

%% Таблицы
\usepackage{array, tabularx, tabulary, booktabs}
\usepackage{longtable}  % Длинные таблицы
\usepackage{multirow} % Слияние строк в таблице

%% Cписки
\usepackage{multicol}
\usepackage{enumitem}

%% Гиперссылки
\usepackage{xcolor}
\usepackage{hyperref}
\definecolor{linkcolor}{HTML}{8b00ff}
\hypersetup{colorlinks = true,
			linkcolor = linkcolor,
			urlcolor = linkcolor,
			citecolor = linkcolor}

%% Выравнивание
\setlength{\parskip}{0.5em} % Расстояние между абзацами
\usepackage{geometry} % Поля
\geometry{
	a4paper,
	left=20mm,
	top=20mm,
	right=20mm}
\setlength{\parindent}{0cm} % Отступ (красная строка)
\linespread{1.0} % Интерлиньяж
\usepackage[many]{tcolorbox}  

%% Оформление

\newtcolorbox{rulesbox}[1]{%
	tikznode boxed title,
	enhanced,
	arc=0mm,
	interior style={white},
	attach boxed title to top center= {yshift=-\tcboxedtitleheight/2},
	fonttitle=\bfseries,
	colbacktitle=white,coltitle=black,
	boxed title style={size=normal,colframe=white,boxrule=0pt},
	title={#1}}

% Красивый серый фон
\usepackage{framed} 
\definecolor{shadecolor}{gray}{0.9}

% Код
\newcommand{\code}[1]{{\tt #1}}

% Колонтитулы
%\usepackage{fancyhdr}
%\pagestyle{fancy}
%\fancyhf{}
%\fancyhead[L]{}
%\fancyhead[R]{\thepage}

% Разделы и подразделы
\usepackage[sf, sl, outermarks]{titlesec}
\titleformat{\section}{\Large\bfseries\sffamily}{\thesection}{0.5em}{}
\titleformat{\subsection}{\large\sffamily}{\thesubsection}{0.5em}{}

% Содержание
%\usepackage{tocloft}
%\renewcommand{\cftsecfont}{\hspace{4.5em}\normalfont}
%\renewcommand{\cftsubsecfont}{\hspace{5em}\normalfont}
%\renewcommand{\cftsecpagefont}{\normalfont\hfill}
%\renewcommand{\cfttoctitlefont}{\large\normalfont\hfill}
%\renewcommand{\cftaftertoctitle}{\hfill}
%\renewcommand{\cftsecleader}{\cftdotfill{\cftdotsep}}
%\renewcommand{\cftsecafterpnum}{\hspace*{5.5em}\hfill}
%\renewcommand{\cftsubsecafterpnum}{\hspace*{5.5em}\hfill}
%\renewcommand{\cftsecaftersnum}{.}
%\renewcommand{\cftsubsecaftersnum}{.}

%% Комментарии
\usepackage{comment}

%% To-do
\usepackage{todonotes}

%% Литература
\usepackage[backend = biber,
			bibencoding = utf8, 
			sorting = nty, 
			maxcitenames = 4,
			style = numeric-verb]{biblatex}
\addbibresource{lit.bib}
\usepackage{csquotes}

%% Заголовок
\title{{\normalsize Прикладная статистика в машинном обучении} \\ \vspace{0.5em} Задачи для подготовки к квизу \#4}
\author{\rule{15cm}{0.4pt}}


\begin{document}
	
	\maketitle
	
	\begin{center}
		В каждом вопросе выберите все верные ответы.
	\end{center}
	
	Рассмотрим случайное наблюдение $X$ из нормального распределения, $X \sim \mathcal{N}(\theta, \sigma^2)$. Для простоты предположим, что $\sigma^2 = 5$. Пусть априорное распределение $\theta$ является нормальным $\mathcal{N}(4, 8)$. 
	
	\begin{questions}
		\question На основе условия задачи можно сделать вывод, что
		\begin{choices}
			\choice $f(\theta) = ce^{-(X-\theta)^2/10}$.
			\CorrectChoice $f(\theta) = ce^{-(\theta-4)^2/16}$.
			\choice $f(X | \theta) = ce^{-(\theta-4)^2/16}$.
			\choice $f(X|\theta) = ce^{-\frac{1}{2}(X-\theta)^2/10}$.
			\choice Нет верного ответа.
		\end{choices}
	
		\question Пусть $X = 3$. Апостериорное распределение параметра $\theta$ задаётся как
		\begin{choices}
			\CorrectChoice $f(\theta | X) = Ce^{-\frac{(\theta - 4)^2}{16} - \frac{(3-\theta)^2}{10}}$.
			\choice $f(X | \theta) = Ce^{-\frac{(\theta - 4)^2}{10} - \frac{(3-\theta)^2}{16}}$.
			\choice $f(X, \theta) = Ce^{-\frac{(\theta - 4)^2}{16} - \frac{(3-\theta)^2}{10}}$.
			\choice $f(\theta | X) = Ce^{-\frac{(\theta - 4)^2}{10} - \frac{(3-\theta)^2}{16}}$.
			\choice Нет верного ответа.
		\end{choices}
		
		\question При выделении полного квадрата относительно $\theta$ в степени экспоненты получается выражение
		\begin{choices}
			\CorrectChoice $-\dfrac{(\theta - 44/13)^2 + 152/13 - (44/13)^2}{80/13}$.
			\choice $-\dfrac{(\theta - 126/169)^2 + 4/169 - (28/169)^2}{12/169}$.
			\choice $-\dfrac{(\theta - 4)^2 + 12/13 - (155/13)^2}{1/13}$.
			\choice $-\dfrac{(\theta + 1/21)^2 - 7/21 + (14/21)^2}{78/21}$.
			\choice Нет верного ответа.
		\end{choices}
		
		\question Апостериорное распределение $\theta$ с точностью до константы является
		\begin{choices}
			\choice $\mathrm{Beta}(1, 3)$.
			\choice стандартным нормальным.
			\CorrectChoice нормальным.
			\choice распределением Вейбулла.
			\choice Нет верного ответа.
		\end{choices}
	
		\question 95\%-ый байесовский доверительный интервал для $\theta$
		\begin{choices}
			\choice совпадает с частотным.
			\choice совпадает с 10\%-ым байесовским доверительным интервалом для $\sigma^2$.
			\choice не может быть приближен при помощи симуляции
			\choice вычисляется как $\P(\theta \in (a, b) | X) < 0.95$.
			 \CorrectChoice Нет верного ответа.
		\end{choices}
	\end{questions}

	Пусть $X_1$, $\ldots$, $X_n$ -- выборка независимых одинаково распределённых случайных величин из нормального распределения $\mathcal{N}(\mu, 1)$.
	
	\begin{questions}
		\question При проверке гипотезы $H_0: \mu = 0$ против $H_1: \mu \ne 0$ рассчитывается статистика $(\bar{X} - 0)/\mathrm{se}(\bar{X})$. Эта статистика имеет распределение
		\begin{choices}
			\choice $t_{n-1}$.
			\choice Нормальное.
			\choice Фишера.
			\choice Асимптотически нормальное, но $t$-распределение при малых $n$.
			\choice Нет верного ответа.
		\end{choices}
	
		\question При проверке гипотезы $H_0: \mu = 0$ против $H_1: \mu \ne 0$ оказалось, что p-value равно 0.06. Это означает, что
		\begin{choices}
			\choice нулевая гипотеза отвергается на уровне значимости 5\%.
			\choice нулевая гипотеза отвергается на уровне значимости 1\%.
			\choice нулевая гипотеза не отвергается при любом разумном уровне значимости.
			\choice нулевая гипотеза не отвергается на уровне значимости 1\%.
			\choice Нет верного ответа.
		\end{choices}
	
		\question При проверке гипотезы $H_0: \mu = 0$ против $H_1: \mu \ne 0$
		\begin{choices}
			\item p-value окажется меньше нуля.
			\item если p-value окажется равным 0.55, то $H_0$ будет отвергнута на любом разумном уровне значимости.
			\item если p-value окажется равным 0.000, то $H_0$ будет отвергнута на любом разумном уровне значимости.
			\item p-value равно функции правдоподобия, взятой в точке разницы между истинным параметром и ML-оценкой $\mu$.
			\item Нет верного ответа.
		\end{choices}
	\end{questions}

	При тестировании трёх видов лекарств против плацебо ($H_0: p_i = p_{plac}$) оказалось, что p-value в соответствующих тестах равны $0.000$, $0.001$, $0.08$.
	
	\begin{questions}
		\question На основании условия задачи можно сделать вывод о том, что на уровне значимости 5\%
		\begin{choices}
			\choice только одно лекарство статистически отличается от плацебо.
			\choice только одно лекарство статистически не отличается от плацебо.
			\choice существует разумный уровень значимости, на котором $H_0$ не будет отвергнута для первого лекарства.
			\choice не существует разумного уровня значимости, на котором $H_0$ будет отвергнута для третьего лекарства.
			\choice Нет верного ответа.
		\end{choices}
	
		\question При проведении множественного тестирования методом Бонферрони
		\begin{choices}
			\choice $H_0$ будет отвергнута для третьего лекарства.
			\choice каждое p-value необходимо сравнивать с $\alpha / 2$.
			\choice только одна нулевая гипотеза не будет отвергнута.
			\choice следует всегда выбрать уровень значимости 10\%.
			\choice Нет верного ответа.
		\end{choices}
	\end{questions}

	\newpage
	
	Ответы:
	\begin{enumerate}
		\item B.
		\item A.
		\item A.
		\item C.
		\item E.
		\item B.
		\item D.
		\item C.
		\item B.
		\item C.
	\end{enumerate}

\end{document} 
	