\documentclass[10pt, a4paper]{exam}

%% Язык
\usepackage{cmap} % Поиск в PDF
\usepackage{mathtext} % Кириллица в формулах
\usepackage[T2A]{fontenc} % Кодировка
\usepackage[utf8]{inputenc} % Кодировка
\usepackage[english,russian]{babel} % Локализация, переносы

%% Шрифты

% Serif
%\usepackage{euscript} % Шрифт Евклид
%\usepackage{mathrsfs} % Шрифт для математики
\usepackage{libertinus}

% Sans-serif
%\renewcommand{\rmdefault}{cmss}
%\renewcommand{\ttdefault}{cmss}
%\usepackage{sfmath}

% Настройки для xelatex
%\usepackage{polyglossia} % Для выбора языка в xelatex
%\setmainlanguage{russian}
%\setotherlanguages{english}
% Ligatures=TeX is on by default
% https://tex.stackexchange.com/questions/323542/
%\setmainfont[Ligatures=TeX]{Cantarell}
%\newfontfamily{\cyrillicfonttt}{Times New Roman}
%\newfontfamily\cyrillicfont{Cantarell}[Script=Cyrillic]
%\setsansfont[Ligatures=TeX]{Cantarell}
%\newfontfamily\cyrillicfontsf{Cantarell}[Script=Cyrillic]
%\setmonofont{Courier New}
%\newfontfamily\cyrillicfonttt{Courier New}[Script=Cyrillic]

%% Математика
\usepackage{amsmath, amsfonts, amssymb, amsthm, mathtools}
\usepackage{icomma}

% Операторы
\DeclareMathOperator*\plim{plim}
\DeclareMathOperator{\sgn}{sign}
\DeclareMathOperator{\sign}{sign}
\DeclareMathOperator*{\argmin}{arg\,min}
\DeclareMathOperator*{\argmax}{arg\,max}
\DeclareMathOperator*{\amn}{arg\,min}
\DeclareMathOperator*{\amx}{arg\,max}
\DeclareMathOperator{\cov}{Cov}
\DeclareMathOperator{\Var}{Var}
\DeclareMathOperator{\Cov}{Cov}
\DeclareMathOperator{\Corr}{Corr}
\DeclareMathOperator{\pCorr}{pCorr}
\DeclareMathOperator{\E}{\mathbb{E}}
\let\P\relax
\DeclareMathOperator{\P}{\mathbb{P}}
\renewcommand{\le}{\leqslant}
\renewcommand{\ge}{\geqslant}
\renewcommand{\leq}{\leqslant}
\renewcommand{\geq}{\geqslant}

% Распределения
\newcommand{\cN}{\mathcal{N}}
\newcommand{\cU}{\mathcal{U}}
\newcommand{\cBinom}{\mathcal{Binom}}
\newcommand{\cPois}{\mathcal{Pois}}
\newcommand{\cBeta}{\mathcal{Beta}}
\newcommand{\cGamma}{\mathcal{Gamma}}

% Множества
\def \R{\mathbb{R}}
\def \N{\mathbb{N}}
\def \Z{\mathbb{Z}}

% Другое
\newcommand{\dx}[1]{\,\mathrm{d}#1} % Для интеграла: маленький отступ и прямая d
\newcommand{\ind}[1]{\mathbbm{1}_{\{#1\}}} % Индикатор события
\newcommand{\iid}{\mathrel{\stackrel{\rm i.\,i.\,d.}\sim}}
\newcommand{\const}{\mathrm{const}}

%% Изображения
\usepackage{graphicx}
\usepackage{caption}
\usepackage{subcaption}
\usepackage{physics}
\usepackage{wrapfig} % Обтекание рисунков и таблиц текстом
\usepackage{tikz}

%% Таблицы
\usepackage{array, tabularx, tabulary, booktabs}
\usepackage{longtable}  % Длинные таблицы
\usepackage{multirow} % Слияние строк в таблице

%% Cписки
\usepackage{multicol}
\usepackage{enumitem}

%% Гиперссылки
\usepackage{xcolor}
\usepackage{hyperref}
\definecolor{linkcolor}{HTML}{8b00ff}
\hypersetup{colorlinks = true,
			linkcolor = linkcolor,
			urlcolor = linkcolor,
			citecolor = linkcolor}

%% Выравнивание
\setlength{\parskip}{0.5em} % Расстояние между абзацами
\usepackage{geometry} % Поля
\geometry{
	a4paper,
	left=20mm,
	top=20mm,
	right=20mm}
\setlength{\parindent}{0cm} % Отступ (красная строка)
\linespread{1.0} % Интерлиньяж
\usepackage[many]{tcolorbox}  

%% Оформление

\newtcolorbox{rulesbox}[1]{%
	tikznode boxed title,
	enhanced,
	arc=0mm,
	interior style={white},
	attach boxed title to top center= {yshift=-\tcboxedtitleheight/2},
	fonttitle=\bfseries,
	colbacktitle=white,coltitle=black,
	boxed title style={size=normal,colframe=white,boxrule=0pt},
	title={#1}}

% Красивый серый фон
\usepackage{framed} 
\definecolor{shadecolor}{gray}{0.9}

% Код
\newcommand{\code}[1]{{\tt #1}}

% Колонтитулы
%\usepackage{fancyhdr}
%\pagestyle{fancy}
%\fancyhf{}
%\fancyhead[L]{}
%\fancyhead[R]{\thepage}

% Разделы и подразделы
\usepackage[sf, sl, outermarks]{titlesec}
\titleformat{\section}{\Large\bfseries\sffamily}{\thesection}{0.5em}{}
\titleformat{\subsection}{\large\sffamily}{\thesubsection}{0.5em}{}

% Содержание
%\usepackage{tocloft}
%\renewcommand{\cftsecfont}{\hspace{4.5em}\normalfont}
%\renewcommand{\cftsubsecfont}{\hspace{5em}\normalfont}
%\renewcommand{\cftsecpagefont}{\normalfont\hfill}
%\renewcommand{\cfttoctitlefont}{\large\normalfont\hfill}
%\renewcommand{\cftaftertoctitle}{\hfill}
%\renewcommand{\cftsecleader}{\cftdotfill{\cftdotsep}}
%\renewcommand{\cftsecafterpnum}{\hspace*{5.5em}\hfill}
%\renewcommand{\cftsubsecafterpnum}{\hspace*{5.5em}\hfill}
%\renewcommand{\cftsecaftersnum}{.}
%\renewcommand{\cftsubsecaftersnum}{.}

%% Комментарии
\usepackage{comment}

%% To-do
\usepackage{todonotes}

%% Литература
\usepackage[backend = biber,
			bibencoding = utf8, 
			sorting = nty, 
			maxcitenames = 4,
			style = numeric-verb]{biblatex}
\addbibresource{lit.bib}
\usepackage{csquotes}

%% Заголовок
\title{{\normalsize Прикладная статистика в машинном обучении} \\ \vspace{0.5em} Квиз \#4D}
\author{\rule{15cm}{0.4pt}}


\begin{document}
	
	\maketitle
	
	\begin{center}
		В каждом вопросе выберите все верные ответы.
	\end{center}
	
	Рассмотрим выборку независимых одинаково распределённых случайных величин $X_1$, $X_2$, $\ldots$, $X_N$ из нормального распределения $\mathcal{N}(\mu, 15)$. Предположим, что априорное распределение $\mu$ является нормальным $\mathcal{N}(1, 1)$.
	
	\begin{questions}
		\question На основе условия задачи можно сделать вывод, что
		\begin{choices} 
			\choice $f(\mu | X) = \prod_i ce^{-\frac{(\mu - 1)^2}{2}}$.
			\choice $f(X | \mu) = \prod_i ce^{-\frac{(X_i - \mu)^2}{15}}$.
			\choice $f(X | \mu) =  ce^{-\frac{(X_i - \mu)^2}{2}}$.
			\choice $f(\mu | X) = ce^{-\frac{(\mu - 1)^2}{2}}$.
			\choice Нет верного ответа.
		\end{choices}
	
		\question Для простоты далее рассмотрим только наблюдение $X_1$. Оказалось, что $X_1 = 7$. Апостериорное распределение параметра $\mu$ задаётся как
		\begin{choices}
			\choice $f( X_{1} | \mu) = Ce^{\frac{(7 - \mu)^2}{30} + \frac{(\mu - 7)^2}{30}}$.
			\choice $f(\mu | X_{1}) = Ce^{-\frac{(7 - \mu)^2}{30} - \frac{(\mu - 1)^2}{2}}$.
			\choice $f(\mu | X_{1}) = Ce^{-\frac{(7 - \mu)^2}{2} - \frac{(\mu - 7)^2}{30}}$.
			\choice $f(X_1 | \mu) = Ce^{-\frac{\mu^2}{30} + \frac{(7 - \mu)^2}{60}}$.
			\choice Нет верного ответа.
		\end{choices}
	
		\question Апостериорное распределение $\mu$ с точностью до константы является 
		\begin{choices}
			\choice Стандартным нормальным.
			\choice Имеющим конечное математическое ожидание.
			\choice Экспоненциальным распределением.
			\choice Нормальным, имеющим бесконечную дисперсию.
			\choice Нет верного ответа.
		\end{choices}
		
		\question Константа $C$
		\begin{choices}
			\choice Обычно принимается равной $\bar{X}$.
			\choice Не вычисляется даже приблизительно.
			\choice Иногда вычисляется путём угадывания вида апостериорного распределения.
			\choice Обычно получается отрицательной.
			\choice Нет верного ответа.
		\end{choices}
	
		\question Выражение $\P(\mu \in (c, d) | X_5) = 0.9$
		\begin{choices}
			\choice Является формулой $95\%$-го байесовского доверительного интервала.
			\choice Является формулой $90\%$-го частотного доверительного интервала.
			\choice Не может быть вычислено за конечное число итераций. 
			\choice Может быть вычислено при помощи симуляций. 
			\choice Нет верного ответа.
		\end{choices}
	
		\question Точечная байесовская оценка $\mu$
		\begin{choices}
			\choice Может быть получена из анализа гистограммы апостериорного распределения.
			\choice Равна медиане $f(\mu)$.
			\choice Всегда совпадает с точечной частотной оценкой $\mu$.
			\choice Не может быть равна моде апостериорного распределения.
			\choice Нет верного ответа.
		\end{choices}
		\vspace{1em}
	
		Далее будем рассуждать в терминах частотного подхода и считать, что $\mu$ -- константа.
		\vspace{1em}
		
		\question Пусть тестируется гипотеза $H_0: \mu = 20$ против $H_1: \mu > 20$. Тогда
		\begin{choices}
			\choice Если p-value окажется равным 0.01, то 10\%-ом уровне значимости тест Вальда отвергнет нулевую гипотезу. 
			\choice Если p-value окажется равным 0.01, то 10\%-ом уровне значимости LR-тест не отвергнет нулевую гипотезу. 
			\choice Если при использовании Z-теста p-value окажется 0.99, то $H_0$ будет отвергнута на любом разумном уровне значимости.
			\choice Если используется LR-тест, но p-value не существует.
			\choice Нет верного ответа.
		\end{choices}
	
		\question Пусть тестируется гипотеза $H_0: \mu = 12$ против $H_1: \mu \ne 12$. Тогда
		\begin{choices}
			\choice p-value обязательно лежит в границах $[0.000, 0.12]$.
			\choice Если p-value равно 0.5, то нулевая гипотеза не отвергается на уровне значимости 51\%.
			\choice p-value обязательно лежит в границах $[0.000, 1.000]$.
			\choice Если p-value равно 0.000, то существует разумный уровень значимости, на котором нулевая гипотеза не отвергается.
			\choice Нет верного ответа.
		\end{choices}
	\end{questions}
	
	При тестировании трёх видов лекарств против плацебо ($H_{0,i}: p_i = p_{plac}$) оказалось, что соответствующие p-value равны 1.000, 1.000, 0.000.
	
	\begin{questions}
		\question На основании условия задачи можно сделать вывод, что на уровне значимости 5\%
		\begin{choices}
			\item Только третье лекарство статистически неотличимо от плацебо.
			\item Первое лекарство статистически отлично от плацебо на уровне значимости $10\%$.
			\item Все лекарства статистически неотличимы от плацебо.
			\item Только первые два лекарства статистически отличны от плацебо.
			\item Нет верного ответа.
		\end{choices}
	
		\question При проведении множественного тестирования методом Бонферрони
		\begin{choices}
			\item Только первое лекарство статистически неотличимо от плацебо.
			\item Только первое лекарство статистически отлично от плацебо.
			\item Только третье лекарство статистически отлично от плацебо.
			\item Первое и второе лекарство статистически отличны от плацебо на уровне значимости 5\%.
			\item Нет верного ответа.
		\end{choices}
	
		\question При проведении множественного тестирования методом Бенджамини-Хохберга
		\begin{choices}
			\item На первом шаге p-value следует упорядочить как 1.000, 1.000, 0.000.
			\item Окажется невозможным сравнить $p_{(1)}$ и $\ell_{(1)}$.
			\item Результаты тестирования ($i$-ая гипотеза отвергается/не отвергается) не будут совпадать с результатами метода Бонферрони.  
			\item Результаты тестирования ($i$-ая гипотеза отвергается/не отвергается) будут совпадать с результатами без корректировки.  
			\item Нет верного ответа.
		\end{choices}
	
		\question При проведении множественного тестирования методов Бенджамини-Хохберга на уровне значимости 5\%
		\begin{choices}
			\item Ровно два лекарства статистически неотличимы от плацебо.
			\item Ровно одно лекарство статистически неотличимо от плацебо.
			\item Все три лекарства статистически неотличимы от плацебо.
			\item Все три лекарства статистически отличны от плацебо.
			\item Нет верного ответа.
		\end{choices}
	\end{questions}
	


\end{document} 
	