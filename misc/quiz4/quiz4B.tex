\documentclass[10pt, a4paper]{exam}

%% Язык
\usepackage{cmap} % Поиск в PDF
\usepackage{mathtext} % Кириллица в формулах
\usepackage[T2A]{fontenc} % Кодировка
\usepackage[utf8]{inputenc} % Кодировка
\usepackage[english,russian]{babel} % Локализация, переносы

%% Шрифты

% Serif
%\usepackage{euscript} % Шрифт Евклид
%\usepackage{mathrsfs} % Шрифт для математики
\usepackage{libertinus}

% Sans-serif
%\renewcommand{\rmdefault}{cmss}
%\renewcommand{\ttdefault}{cmss}
%\usepackage{sfmath}

% Настройки для xelatex
%\usepackage{polyglossia} % Для выбора языка в xelatex
%\setmainlanguage{russian}
%\setotherlanguages{english}
% Ligatures=TeX is on by default
% https://tex.stackexchange.com/questions/323542/
%\setmainfont[Ligatures=TeX]{Cantarell}
%\newfontfamily{\cyrillicfonttt}{Times New Roman}
%\newfontfamily\cyrillicfont{Cantarell}[Script=Cyrillic]
%\setsansfont[Ligatures=TeX]{Cantarell}
%\newfontfamily\cyrillicfontsf{Cantarell}[Script=Cyrillic]
%\setmonofont{Courier New}
%\newfontfamily\cyrillicfonttt{Courier New}[Script=Cyrillic]

%% Математика
\usepackage{amsmath, amsfonts, amssymb, amsthm, mathtools}
\usepackage{icomma}

% Операторы
\DeclareMathOperator*\plim{plim}
\DeclareMathOperator{\sgn}{sign}
\DeclareMathOperator{\sign}{sign}
\DeclareMathOperator*{\argmin}{arg\,min}
\DeclareMathOperator*{\argmax}{arg\,max}
\DeclareMathOperator*{\amn}{arg\,min}
\DeclareMathOperator*{\amx}{arg\,max}
\DeclareMathOperator{\cov}{Cov}
\DeclareMathOperator{\Var}{Var}
\DeclareMathOperator{\Cov}{Cov}
\DeclareMathOperator{\Corr}{Corr}
\DeclareMathOperator{\pCorr}{pCorr}
\DeclareMathOperator{\E}{\mathbb{E}}
\let\P\relax
\DeclareMathOperator{\P}{\mathbb{P}}
\renewcommand{\le}{\leqslant}
\renewcommand{\ge}{\geqslant}
\renewcommand{\leq}{\leqslant}
\renewcommand{\geq}{\geqslant}

% Распределения
\newcommand{\cN}{\mathcal{N}}
\newcommand{\cU}{\mathcal{U}}
\newcommand{\cBinom}{\mathcal{Binom}}
\newcommand{\cPois}{\mathcal{Pois}}
\newcommand{\cBeta}{\mathcal{Beta}}
\newcommand{\cGamma}{\mathcal{Gamma}}

% Множества
\def \R{\mathbb{R}}
\def \N{\mathbb{N}}
\def \Z{\mathbb{Z}}

% Другое
\newcommand{\dx}[1]{\,\mathrm{d}#1} % Для интеграла: маленький отступ и прямая d
\newcommand{\ind}[1]{\mathbbm{1}_{\{#1\}}} % Индикатор события
\newcommand{\iid}{\mathrel{\stackrel{\rm i.\,i.\,d.}\sim}}
\newcommand{\const}{\mathrm{const}}

%% Изображения
\usepackage{graphicx}
\usepackage{caption}
\usepackage{subcaption}
\usepackage{physics}
\usepackage{wrapfig} % Обтекание рисунков и таблиц текстом
\usepackage{tikz}

%% Таблицы
\usepackage{array, tabularx, tabulary, booktabs}
\usepackage{longtable}  % Длинные таблицы
\usepackage{multirow} % Слияние строк в таблице

%% Cписки
\usepackage{multicol}
\usepackage{enumitem}

%% Гиперссылки
\usepackage{xcolor}
\usepackage{hyperref}
\definecolor{linkcolor}{HTML}{8b00ff}
\hypersetup{colorlinks = true,
			linkcolor = linkcolor,
			urlcolor = linkcolor,
			citecolor = linkcolor}

%% Выравнивание
\setlength{\parskip}{0.5em} % Расстояние между абзацами
\usepackage{geometry} % Поля
\geometry{
	a4paper,
	left=20mm,
	top=20mm,
	right=20mm}
\setlength{\parindent}{0cm} % Отступ (красная строка)
\linespread{1.0} % Интерлиньяж
\usepackage[many]{tcolorbox}  

%% Оформление

\newtcolorbox{rulesbox}[1]{%
	tikznode boxed title,
	enhanced,
	arc=0mm,
	interior style={white},
	attach boxed title to top center= {yshift=-\tcboxedtitleheight/2},
	fonttitle=\bfseries,
	colbacktitle=white,coltitle=black,
	boxed title style={size=normal,colframe=white,boxrule=0pt},
	title={#1}}

% Красивый серый фон
\usepackage{framed} 
\definecolor{shadecolor}{gray}{0.9}

% Код
\newcommand{\code}[1]{{\tt #1}}

% Колонтитулы
%\usepackage{fancyhdr}
%\pagestyle{fancy}
%\fancyhf{}
%\fancyhead[L]{}
%\fancyhead[R]{\thepage}

% Разделы и подразделы
\usepackage[sf, sl, outermarks]{titlesec}
\titleformat{\section}{\Large\bfseries\sffamily}{\thesection}{0.5em}{}
\titleformat{\subsection}{\large\sffamily}{\thesubsection}{0.5em}{}

% Содержание
%\usepackage{tocloft}
%\renewcommand{\cftsecfont}{\hspace{4.5em}\normalfont}
%\renewcommand{\cftsubsecfont}{\hspace{5em}\normalfont}
%\renewcommand{\cftsecpagefont}{\normalfont\hfill}
%\renewcommand{\cfttoctitlefont}{\large\normalfont\hfill}
%\renewcommand{\cftaftertoctitle}{\hfill}
%\renewcommand{\cftsecleader}{\cftdotfill{\cftdotsep}}
%\renewcommand{\cftsecafterpnum}{\hspace*{5.5em}\hfill}
%\renewcommand{\cftsubsecafterpnum}{\hspace*{5.5em}\hfill}
%\renewcommand{\cftsecaftersnum}{.}
%\renewcommand{\cftsubsecaftersnum}{.}

%% Комментарии
\usepackage{comment}

%% To-do
\usepackage{todonotes}

%% Литература
\usepackage[backend = biber,
			bibencoding = utf8, 
			sorting = nty, 
			maxcitenames = 4,
			style = numeric-verb]{biblatex}
\addbibresource{lit.bib}
\usepackage{csquotes}

%% Заголовок
\title{{\normalsize Прикладная статистика в машинном обучении} \\ \vspace{0.5em} Квиз \#4Б}
\author{\rule{15cm}{0.4pt}}


\begin{document}
	
	\maketitle
	
	\begin{center}
		В каждом вопросе выберите все верные ответы.
	\end{center}
	
	Рассмотрим выборку независимых одинаково распределённых случайных величин $X_1$, $X_2$, $\ldots$, $X_N$ из нормального распределения $\mathcal{N}(\mu, 2)$. Предположим, что априорное распределение $\mu$ является нормальным $\mathcal{N}(3, 6)$.
	
	\begin{questions}
		\question На основе условия задачи можно сделать вывод, что
		\begin{choices}
			\choice $f(X | \mu) = \prod_i ce^{-\frac{(X_i - \mu)^2}{4}}$.
			\choice $f(\mu | X) = \prod_i ce^{-\frac{(X_i - \mu)^2}{4}}$.
			\choice $f(X | \mu) = ce^{-\frac{(X_i - \mu)^2}{4}}$.
			\choice $f(\mu | X) = ce^{-\frac{(\mu - 3)^2}{2}}$.
			\choice Нет верного ответа.
		\end{choices}
	
		\question Для простоты далее рассмотрим только наблюдение $X_5$. Оказалось, что $X_5 = 100$. Апостериорное распределение параметра $\mu$ задаётся как
		\begin{choices}
			\choice $f(\mu | X_{5}) = \prod_i Ce^{\frac{(100 - \mu)^2}{8}  \frac{(\mu - 3)^2}{8}}$.
			\choice$f(\mu | X_{5}) = Ce^{-\frac{(100 - \mu)^2}{4} - \frac{(\mu - 3)^2}{12}}$.
			\choice $f(X_5 | \mu) = Ce^{-\frac{\mu^2}{2} + \frac{(100 - \mu)^2}{8}}$.
			\choice$f(\mu | X_{5}) = Ce^{-\frac{(100 - \mu)^2}{4} - \frac{(\mu - 3)^2}{12}}$.
			\choice Нет верного ответа.
		\end{choices}
	
		\question Апостериорное распределение $\mu$ с точностью до константы является 
		\begin{choices}
			\choice Экспоненциальным.
			\choice $t$-распределением.
			\choice Распределением Пуассона с $\lambda = \mu$.
			\choice Нормальным и имеющим конечное математическое ожидание.
			\choice Нет верного ответа.
		\end{choices}
		
		\question Константа $C$
		\begin{choices}
			\choice Всегда окажется положительной.
			\choice Не вычисляется в явном виде.
			\choice Не вычисляется даже приблизительно.
			\choice Всегда является отрицательной.
			\choice Нет верного ответа.
		\end{choices}
	
		\question Выражение $\P(\mu \in (c, d) | X_5) = 0.99$
		\begin{choices}
			\choice Является формулой $95\%$-го байесовского доверительного интервала.
			\choice Не может быть вычислено аналитически.
			\choice Не имеет смысла в частотном подходе.
			\choice Равно нулю, если $\mu$ -- дискретная случайная величина.
			\choice Нет верного ответа.
		\end{choices}
	
		\question Точечная байесовская оценка $\mu$
		\begin{choices}
			\choice Совпадает с интервальной оценкой.
			\choice Не может быть получена аналитически.
			\choice Находится из задачи максимизации правдоподобия априорного распределения $\mu$.
			\choice Не может быть равна моде апостериорного распределения $\mu$. 
			\choice Нет верного ответа.
		\end{choices}
		\vspace{1em}
	
		Далее будем рассуждать в терминах частотного подхода и считать, что $\mu$ -- константа.
		\vspace{1em}
		
		\question Пусть тестируется гипотеза $H_0: \mu = 4$ против $H_1: \mu > 4$. Тогда
		\begin{choices}
			\choice p-value не существует.
			\choice p-value совпадает с производной функции правдоподобия в точке $4$.
			\choice p-value не может быть больше $1$.
			\choice p-value может быть отрицательным.
			\choice Нет верного ответа.
		\end{choices}
	
		\question Пусть тестируется гипотеза $H_0: \mu = 99$ против $H_1: \mu \ne 99$. Тогда
		\begin{choices}
			\choice При использовании $Z$-теста p-value равно 0.99.
			\choice При использовании LM-теста p-value получится близким к нулю.
			\choice При использовании LR-теста p-value не будет существовать с вероятностью 1\%.
			\choice Если p-value окажется близким к 0, то нулевая гипотеза будет отвергнута на уровне значимости 99\%. 
			\choice Нет верного ответа.
		\end{choices}
	\end{questions}
	
	При тестировании трёх видов лекарств против плацебо ($H_{0,i}: p_i = p_{plac}$) оказалось, что соответствующие p-value равны 0.000, 0.000, 0.89.
	
	\begin{questions}
		\question На основании условия задачи можно сделать вывод, что на уровне значимости 5\%
		\begin{choices}
			\item Только третье лекарство статистически неотличимо от плацебо.
			\item Если $Z$-статистика для первого лекарства меньше 0, то первое лекарство статистически неотличимо от плацебо.
			\item Все лекарства статистически неотличимы от плацебо.
			\item Первое лекарство статистически неотличимо от плацебо на уровне значимости $10\%$.
			\item Нет верного ответа.
		\end{choices}
	
		\question При проведении множественного тестирования методом Бонферрони
		\begin{choices}
			\item Только первое лекарство статистически неотличимо от плацебо.
			\item Результат тестирования ($i$-ая гипотеза отвергается/не отвергается) совпадёт с результатом без корректировки.
			\item Уровень значимости следует принять равным $10\%$.
			\item Не существует такого разумного уровня значимости, на котором первое лекарство было бы статистически отличимо от плацебо.
			\item Нет верного ответа.
		\end{choices}
	
		\question При проведении множественного тестирования методом Бенджамини-Хохберга
		\begin{choices}
			\item На первом шаге p-value следует упорядочить по возрастанию и исключить медиану полученного вариационного ряда.
			\item Финальный rejection threshold совпадёт с первой p-value в упорядоченном по возрастании ряду p-value.
			\item Финальный rejection threshold окажется равным уровню значимости.
			\item Результаты тестирования совпадут с результатами метода Бонеферрони.
			\item Нет верного ответа.
		\end{choices}
	
		\question При проведении множественного тестирования методов Бенджамини-Хохберга на уровне значимости 5\%
		\begin{choices}
			\item Ровно два лекарства статистически неотличимы от плацебо.
			\item Ровно одно лекарство статистически неотличимо от плацебо.
			\item Все три лекарства статистически неотличимы от плацебо.
			\item Все три лекарства статистически отличны от плацебо.
			\item Нет верного ответа.
		\end{choices}
	\end{questions}
	


\end{document} 
	