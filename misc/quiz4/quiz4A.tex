\documentclass[10pt, a4paper]{exam}

%% Язык
\usepackage{cmap} % Поиск в PDF
\usepackage{mathtext} % Кириллица в формулах
\usepackage[T2A]{fontenc} % Кодировка
\usepackage[utf8]{inputenc} % Кодировка
\usepackage[english,russian]{babel} % Локализация, переносы

%% Шрифты

% Serif
%\usepackage{euscript} % Шрифт Евклид
%\usepackage{mathrsfs} % Шрифт для математики
\usepackage{libertinus}

% Sans-serif
%\renewcommand{\rmdefault}{cmss}
%\renewcommand{\ttdefault}{cmss}
%\usepackage{sfmath}

% Настройки для xelatex
%\usepackage{polyglossia} % Для выбора языка в xelatex
%\setmainlanguage{russian}
%\setotherlanguages{english}
% Ligatures=TeX is on by default
% https://tex.stackexchange.com/questions/323542/
%\setmainfont[Ligatures=TeX]{Cantarell}
%\newfontfamily{\cyrillicfonttt}{Times New Roman}
%\newfontfamily\cyrillicfont{Cantarell}[Script=Cyrillic]
%\setsansfont[Ligatures=TeX]{Cantarell}
%\newfontfamily\cyrillicfontsf{Cantarell}[Script=Cyrillic]
%\setmonofont{Courier New}
%\newfontfamily\cyrillicfonttt{Courier New}[Script=Cyrillic]

%% Математика
\usepackage{amsmath, amsfonts, amssymb, amsthm, mathtools}
\usepackage{icomma}

% Операторы
\DeclareMathOperator*\plim{plim}
\DeclareMathOperator{\sgn}{sign}
\DeclareMathOperator{\sign}{sign}
\DeclareMathOperator*{\argmin}{arg\,min}
\DeclareMathOperator*{\argmax}{arg\,max}
\DeclareMathOperator*{\amn}{arg\,min}
\DeclareMathOperator*{\amx}{arg\,max}
\DeclareMathOperator{\cov}{Cov}
\DeclareMathOperator{\Var}{Var}
\DeclareMathOperator{\Cov}{Cov}
\DeclareMathOperator{\Corr}{Corr}
\DeclareMathOperator{\pCorr}{pCorr}
\DeclareMathOperator{\E}{\mathbb{E}}
\let\P\relax
\DeclareMathOperator{\P}{\mathbb{P}}
\renewcommand{\le}{\leqslant}
\renewcommand{\ge}{\geqslant}
\renewcommand{\leq}{\leqslant}
\renewcommand{\geq}{\geqslant}

% Распределения
\newcommand{\cN}{\mathcal{N}}
\newcommand{\cU}{\mathcal{U}}
\newcommand{\cBinom}{\mathcal{Binom}}
\newcommand{\cPois}{\mathcal{Pois}}
\newcommand{\cBeta}{\mathcal{Beta}}
\newcommand{\cGamma}{\mathcal{Gamma}}

% Множества
\def \R{\mathbb{R}}
\def \N{\mathbb{N}}
\def \Z{\mathbb{Z}}

% Другое
\newcommand{\dx}[1]{\,\mathrm{d}#1} % Для интеграла: маленький отступ и прямая d
\newcommand{\ind}[1]{\mathbbm{1}_{\{#1\}}} % Индикатор события
\newcommand{\iid}{\mathrel{\stackrel{\rm i.\,i.\,d.}\sim}}
\newcommand{\const}{\mathrm{const}}

%% Изображения
\usepackage{graphicx}
\usepackage{caption}
\usepackage{subcaption}
\usepackage{physics}
\usepackage{wrapfig} % Обтекание рисунков и таблиц текстом
\usepackage{tikz}

%% Таблицы
\usepackage{array, tabularx, tabulary, booktabs}
\usepackage{longtable}  % Длинные таблицы
\usepackage{multirow} % Слияние строк в таблице

%% Cписки
\usepackage{multicol}
\usepackage{enumitem}

%% Гиперссылки
\usepackage{xcolor}
\usepackage{hyperref}
\definecolor{linkcolor}{HTML}{8b00ff}
\hypersetup{colorlinks = true,
			linkcolor = linkcolor,
			urlcolor = linkcolor,
			citecolor = linkcolor}

%% Выравнивание
\setlength{\parskip}{0.5em} % Расстояние между абзацами
\usepackage{geometry} % Поля
\geometry{
	a4paper,
	left=20mm,
	top=20mm,
	right=20mm}
\setlength{\parindent}{0cm} % Отступ (красная строка)
\linespread{1.0} % Интерлиньяж
\usepackage[many]{tcolorbox}  

%% Оформление

\newtcolorbox{rulesbox}[1]{%
	tikznode boxed title,
	enhanced,
	arc=0mm,
	interior style={white},
	attach boxed title to top center= {yshift=-\tcboxedtitleheight/2},
	fonttitle=\bfseries,
	colbacktitle=white,coltitle=black,
	boxed title style={size=normal,colframe=white,boxrule=0pt},
	title={#1}}

% Красивый серый фон
\usepackage{framed} 
\definecolor{shadecolor}{gray}{0.9}

% Код
\newcommand{\code}[1]{{\tt #1}}

% Колонтитулы
%\usepackage{fancyhdr}
%\pagestyle{fancy}
%\fancyhf{}
%\fancyhead[L]{}
%\fancyhead[R]{\thepage}

% Разделы и подразделы
\usepackage[sf, sl, outermarks]{titlesec}
\titleformat{\section}{\Large\bfseries\sffamily}{\thesection}{0.5em}{}
\titleformat{\subsection}{\large\sffamily}{\thesubsection}{0.5em}{}

% Содержание
%\usepackage{tocloft}
%\renewcommand{\cftsecfont}{\hspace{4.5em}\normalfont}
%\renewcommand{\cftsubsecfont}{\hspace{5em}\normalfont}
%\renewcommand{\cftsecpagefont}{\normalfont\hfill}
%\renewcommand{\cfttoctitlefont}{\large\normalfont\hfill}
%\renewcommand{\cftaftertoctitle}{\hfill}
%\renewcommand{\cftsecleader}{\cftdotfill{\cftdotsep}}
%\renewcommand{\cftsecafterpnum}{\hspace*{5.5em}\hfill}
%\renewcommand{\cftsubsecafterpnum}{\hspace*{5.5em}\hfill}
%\renewcommand{\cftsecaftersnum}{.}
%\renewcommand{\cftsubsecaftersnum}{.}

%% Комментарии
\usepackage{comment}

%% To-do
\usepackage{todonotes}

%% Литература
\usepackage[backend = biber,
			bibencoding = utf8, 
			sorting = nty, 
			maxcitenames = 4,
			style = numeric-verb]{biblatex}
\addbibresource{lit.bib}
\usepackage{csquotes}

%% Заголовок
\title{{\normalsize Прикладная статистика в машинном обучении} \\ \vspace{0.5em} Квиз \#4A}
\author{\rule{15cm}{0.4pt}}


\begin{document}
	
	\maketitle
	
	\begin{center}
		В каждом вопросе выберите все верные ответы.
	\end{center}
	
	Рассмотрим выборку независимых одинаково распределённых случайных величин $X_1$, $X_2$, $\ldots$, $X_N$ из нормального распределения $\mathcal{N}(\mu, 4)$. Предположим, что априорное распределение $\mu$ является нормальным $\mathcal{N}(1, 2)$.
	
	\begin{questions}
		\question На основе условия задачи можно сделать вывод, что
		\begin{choices}
			\choice $f(X | \mu) = ce^{-\frac{(X_1 - \mu)^2}{8}}$.
			\choice $f(\mu | X) = \prod_i ce^{-\frac{(\mu - i)^2}{4}}$.
			\choice $f(X | \mu) = \prod_i ce^{-\frac{(X_i - \mu)^2}{8}}$.
			\choice $f(\mu | X) = ce^{-\frac{\mu^2}{8}}$.
			\choice Нет верного ответа.
		\end{choices}
	
		\question Для простоты далее рассмотрим только наблюдение $X_9$. Оказалось, что $X_9 = 10$. Апостериорное распределение параметра $\mu$ задаётся как
		\begin{choices}
			\choice $f(X_9 | \mu) = Ce^{-\frac{X_9^2}{12} - \frac{(\mu - 1)^2}{24}}$.
			\choice$f(\mu | X_{9}) = Ce^{-\frac{(X_9 - 1)^2}{4} + \frac{(\mu - 1)^2}{8}}$.
			\choice $f(\mu | X_{9}) = Ce^{-\frac{(X_9 - \mu)^2}{8} - \frac{(\mu - 1)^2}{4}}$.
			\choice $f(X_9 | \mu) = Ce^{-\frac{(X_9 - \mu)^2}{12} - \frac{(X_9 - 1)^2}{8}}$.
			\choice Нет верного ответа.
		\end{choices}
	
		\question Апостериорное распределение $\mu$ с точностью до константы является 
		\begin{choices}
			\choice Экспоненциальным.
			\choice $t$-распределением.
			\choice Распределением Пуассона с $\lambda = \mu$.
			\choice Нормальным и имеющим бесконечное математическое ожидание.
			\choice Нет верного ответа.
		\end{choices}
		
		\question Константу $C$
		\begin{choices}
			\choice Невозможно восстановить даже приблизительно.
			\choice Можно восстановить только приблизительно.
			\choice Считают равной $1$.
			\choice Иногда можно восстановить аналитически.
			\choice Нет верного ответа.
		\end{choices}
	
		\question Выражение $\P(\mu \in (c, d) | X_9) = 0.95$
		\begin{choices}
			\choice Бессмысленно для частотного подхода, а потому и для байесовского подхода.
			\choice Используется для построения $95\%$-го частотного доверительного интервала для $\mu$.
			\choice Имеет смысл, только когда $\plim_{n \to \infty} \mu = 1$. 
			\choice Означает, что с $95\%$-ой апостериорной вероятностью $\mu$ лежит в интервале $(c, d)$.
			\choice Нет верного ответа.
		\end{choices}
	
		\question В качестве разумной точечной байесовской оценки $\mu$
		\begin{choices}
			\choice Возможно взять среднее, моду или медиану $f(\mu | X)$.
			\choice Возможно взять среднее $f(X|\mu)$.
			\choice Следует взять $\bar{X}$.
			\choice Следует взять $ML$-оценку математического ожидания $f(\mu)$.
			\choice Нет верного ответа.
		\end{choices}
		\vspace{1em}
	
		Далее будем рассуждать в терминах частотного подхода и считать, что $\mu$ -- константа.
		\vspace{1em}
		
		\question Пусть тестируется гипотеза $H_0: \mu = 0$ против $H_1: \mu < 0$. Тогда
		\begin{choices}
			\choice p-value будет равно $1/2$.
			\choice Если p-value окажется меньше 0, то $H_0$ не будет отвергнута на любом разумном уровне значимости.
			\choice p-value равно $\P\{\hat{\mu} < \mu_{obs} | H_0\}$, где $\mu_{obs}$ -- наблюдаемое значение статистики.
			\choice p-value равно $2\P\{\hat{\mu} < \mu_{obs} | H_0\}$, где $\mu_{obs}$ -- наблюдаемое значение статистики.
			\choice Нет верного ответа.
		\end{choices}
	
		\question Пусть тестируется гипотеза $H_0: \mu = 0$ против $H_1: \mu \ne 0$. Тогда
		\begin{choices}
			\choice p-value возможно рассчитать при использовании теста Вальда, но нельзя при использовании LR-теста.
			\choice Если при использовании Z-теста p-value окажется равным 0.06, то $H_0$ будет отвергнута на $5\%$ уровне значимости.
			\choice При использовании LR-теста p-value всегда будет получаться большим 10\%.
			\choice Если при использовании Z-теста p-value окажется равным 0.06, то $H_0$ будет отвергнута на $10\%$ уровне значимости.
			\choice Нет верного ответа.
		\end{choices}
	\end{questions}
	
	При тестировании трёх видов лекарств против плацебо ($H_{0,i}: p_i = p_{plac}$) оказалось, что соответствующие p-value равны 0.000, 0.06, 0.89.
	
	\begin{questions}
		\question На основании условия задачи можно сделать вывод, что на уровне значимости 5\%
		\begin{choices}
			\item Не существует разумного уровня значимости, на котором бы отвергалась нулевая гипотеза для третьего лекарства.
			\item Нулевая гипотеза не отвергается только для одного лекарства. 
			\item Первое лекарство статистически неотличимо от плацебо на любом разумном уровне значимости.
			\item Второе лекарство статистически неотличимо от плацебо на любом разумном уровне значимости.
			\item Нет верного ответа.
		\end{choices}
	
		\question При проведении множественного тестирования методом Бонферрони
		\begin{choices}
			\item Каждое p-value необходимо разделить на 3.
			\item Каждое p-value необходимо разделить на 4.
			\item Каждое p-value необходимо разделить на 2.
			\item Только одно лекарство окажется статистически неотличимым от плацебо.
			\item Нет верного ответа.
		\end{choices}
	
		\question При проведении множественного тестирования методом Бенджамини-Хохберга
		\begin{choices}
			\item Каждое p-value необходимо сравнивать с $\alpha / 3$.
			\item Все лекарства окажутся статистически неотличимыми от плацебо.
			\item Пороговое значение для отвержения гипотезы зависит от числа проверяемых гипотез.
			\item Результаты тестирования всегда совпадут с результатами метода Бонферрони.
			\item Нет верного ответа.
		\end{choices}
	
		\question При проведении множественного тестирования методов Бенджамини-Хохберга на уровне значимости 5\%
		\begin{choices}
			\item Ровно два лекарства статистически неотличимы от плацебо.
			\item Ровно одно лекарство статистически неотличимо от плацебо.
			\item Все три лекарства статистически неотличимы от плацебо.
			\item Все три лекарства статистически отличны от плацебо.
			\item Нет верного ответа.
		\end{choices}
	\end{questions}
	


\end{document} 
	