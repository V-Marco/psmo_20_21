\documentclass[10pt, a4paper]{exam}

%% Язык
\usepackage{cmap} % Поиск в PDF
\usepackage{mathtext} % Кириллица в формулах
\usepackage[T2A]{fontenc} % Кодировка
\usepackage[utf8]{inputenc} % Кодировка
\usepackage[english,russian]{babel} % Локализация, переносы

%% Шрифты

% Serif
%\usepackage{euscript} % Шрифт Евклид
%\usepackage{mathrsfs} % Шрифт для математики
\usepackage{libertinus}

% Sans-serif
%\renewcommand{\rmdefault}{cmss}
%\renewcommand{\ttdefault}{cmss}
%\usepackage{sfmath}

% Настройки для xelatex
%\usepackage{polyglossia} % Для выбора языка в xelatex
%\setmainlanguage{russian}
%\setotherlanguages{english}
% Ligatures=TeX is on by default
% https://tex.stackexchange.com/questions/323542/
%\setmainfont[Ligatures=TeX]{Cantarell}
%\newfontfamily{\cyrillicfonttt}{Times New Roman}
%\newfontfamily\cyrillicfont{Cantarell}[Script=Cyrillic]
%\setsansfont[Ligatures=TeX]{Cantarell}
%\newfontfamily\cyrillicfontsf{Cantarell}[Script=Cyrillic]
%\setmonofont{Courier New}
%\newfontfamily\cyrillicfonttt{Courier New}[Script=Cyrillic]

%% Математика
\usepackage{amsmath, amsfonts, amssymb, amsthm, mathtools}
\usepackage{icomma}

% Операторы
\DeclareMathOperator*\plim{plim}
\DeclareMathOperator{\sgn}{sign}
\DeclareMathOperator{\sign}{sign}
\DeclareMathOperator*{\argmin}{arg\,min}
\DeclareMathOperator*{\argmax}{arg\,max}
\DeclareMathOperator*{\amn}{arg\,min}
\DeclareMathOperator*{\amx}{arg\,max}
\DeclareMathOperator{\cov}{Cov}
\DeclareMathOperator{\Var}{Var}
\DeclareMathOperator{\Cov}{Cov}
\DeclareMathOperator{\Corr}{Corr}
\DeclareMathOperator{\pCorr}{pCorr}
\DeclareMathOperator{\E}{\mathbb{E}}
\let\P\relax
\DeclareMathOperator{\P}{\mathbb{P}}
\renewcommand{\le}{\leqslant}
\renewcommand{\ge}{\geqslant}
\renewcommand{\leq}{\leqslant}
\renewcommand{\geq}{\geqslant}

% Распределения
\newcommand{\cN}{\mathcal{N}}
\newcommand{\cU}{\mathcal{U}}
\newcommand{\cBinom}{\mathcal{Binom}}
\newcommand{\cPois}{\mathcal{Pois}}
\newcommand{\cBeta}{\mathcal{Beta}}
\newcommand{\cGamma}{\mathcal{Gamma}}

% Множества
\def \R{\mathbb{R}}
\def \N{\mathbb{N}}
\def \Z{\mathbb{Z}}

% Другое
\newcommand{\dx}[1]{\,\mathrm{d}#1} % Для интеграла: маленький отступ и прямая d
\newcommand{\ind}[1]{\mathbbm{1}_{\{#1\}}} % Индикатор события
\newcommand{\iid}{\mathrel{\stackrel{\rm i.\,i.\,d.}\sim}}
\newcommand{\const}{\mathrm{const}}

%% Изображения
\usepackage{graphicx}
\usepackage{caption}
\usepackage{subcaption}
\usepackage{physics}
\usepackage{wrapfig} % Обтекание рисунков и таблиц текстом
\usepackage{tikz}

%% Таблицы
\usepackage{array, tabularx, tabulary, booktabs}
\usepackage{longtable}  % Длинные таблицы
\usepackage{multirow} % Слияние строк в таблице

%% Cписки
\usepackage{multicol}
\usepackage{enumitem}

%% Гиперссылки
\usepackage{xcolor}
\usepackage{hyperref}
\definecolor{linkcolor}{HTML}{8b00ff}
\hypersetup{colorlinks = true,
			linkcolor = linkcolor,
			urlcolor = linkcolor,
			citecolor = linkcolor}

%% Выравнивание
\setlength{\parskip}{0.5em} % Расстояние между абзацами
\usepackage{geometry} % Поля
\geometry{
	a4paper,
	left=20mm,
	top=20mm,
	right=20mm}
\setlength{\parindent}{0cm} % Отступ (красная строка)
\linespread{1.0} % Интерлиньяж
\usepackage[many]{tcolorbox}  

%% Оформление

\newtcolorbox{rulesbox}[1]{%
	tikznode boxed title,
	enhanced,
	arc=0mm,
	interior style={white},
	attach boxed title to top center= {yshift=-\tcboxedtitleheight/2},
	fonttitle=\bfseries,
	colbacktitle=white,coltitle=black,
	boxed title style={size=normal,colframe=white,boxrule=0pt},
	title={#1}}

% Красивый серый фон
\usepackage{framed} 
\definecolor{shadecolor}{gray}{0.9}

% Код
\newcommand{\code}[1]{{\tt #1}}

% Колонтитулы
%\usepackage{fancyhdr}
%\pagestyle{fancy}
%\fancyhf{}
%\fancyhead[L]{}
%\fancyhead[R]{\thepage}

% Разделы и подразделы
\usepackage[sf, sl, outermarks]{titlesec}
\titleformat{\section}{\Large\bfseries\sffamily}{\thesection}{0.5em}{}
\titleformat{\subsection}{\large\sffamily}{\thesubsection}{0.5em}{}

% Содержание
%\usepackage{tocloft}
%\renewcommand{\cftsecfont}{\hspace{4.5em}\normalfont}
%\renewcommand{\cftsubsecfont}{\hspace{5em}\normalfont}
%\renewcommand{\cftsecpagefont}{\normalfont\hfill}
%\renewcommand{\cfttoctitlefont}{\large\normalfont\hfill}
%\renewcommand{\cftaftertoctitle}{\hfill}
%\renewcommand{\cftsecleader}{\cftdotfill{\cftdotsep}}
%\renewcommand{\cftsecafterpnum}{\hspace*{5.5em}\hfill}
%\renewcommand{\cftsubsecafterpnum}{\hspace*{5.5em}\hfill}
%\renewcommand{\cftsecaftersnum}{.}
%\renewcommand{\cftsubsecaftersnum}{.}

%% Комментарии
\usepackage{comment}

%% To-do
\usepackage{todonotes}

%% Литература
\usepackage[backend = biber,
			bibencoding = utf8, 
			sorting = nty, 
			maxcitenames = 4,
			style = numeric-verb]{biblatex}
\addbibresource{lit.bib}
\usepackage{csquotes}

%% Заголовок
\title{{\normalsize Прикладная статистика в машинном обучении} \\ \vspace{0.5em} Квиз \#4}
\author{\rule{15cm}{0.4pt}}


\begin{document}
	
	\maketitle
	
	\begin{center}
		В каждом вопросе выберите все верные ответы.
	\end{center}
	
	Рассмотрим выборку независимых одинаково распределённых случайных величин $X_1$, $X_2$, $\ldots$, $X_N$ из нормального распределения $\mathcal{N}(\mu, 9)$. Предположим, что априорное распределение $\mu$ является нормальным $\mathcal{N}(0, 4)$.
	
	\begin{questions}
		\question На основе условия задачи можно сделать вывод, что
		\begin{choices}
			\choice $f(X) = \prod_i ce^{-\frac{(X_i - \mu)^2}{9}}$.
			\choice $f(\mu) = \prod_i ce^{-\frac{(-\mu)^2}{8}}$.
			\choice $f(X | \mu) = \prod_i ce^{-\frac{(X_i - \mu)^2}{9}}$.
			\choice $f(\mu | X) = ce^{-\frac{(- X_i)^2}{18}}$.
			\choice Нет верного ответа.
		\end{choices}
	
		\question Для простоты далее рассмотрим только наблюдение $X_{99}$. Оказалось, что $X_{99} = 5$. Апостериорное распределение параметра $\mu$ задаётся как
		\begin{choices}
			\choice $f(\mu | X_{99}) = Ce^{-\frac{(5 - \mu)^2}{9} - \frac{(\mu)^2}{18}}$.
			\choice $f( X_{99} | \mu) = Ce^{\frac{(- \mu)^2}{4} + \frac{(\mu - 5)^2}{8}}$.
			\choice $f(\mu | X_{99}) = Ce^{-\frac{(- \mu)^2}{8} - \frac{(5 - \mu)^2}{18}}$.
			\choice $f(X_{99} | \mu) = Ce^{-\frac{\mu^2}{9} + \frac{(5 - \mu)^2}{18}}$.
			\choice Нет верного ответа.
		\end{choices}
	
		\question Апостериорное распределение $\mu$ с точностью до константы является 
		\begin{choices}
			\choice Стандартным нормальным.
			\choice Нормальным, имеющим бесконечную дисперсию.
			\choice Вырожденным.
			\choice Распределением Снедекора.
			\choice Нет верного ответа.
		\end{choices}
		
		\question Константа $C$
		\begin{choices}
			\choice Может быть рассчитана путём угадывания вида апостериорного распределения.
			\choice Никогда не вычисляется аналитически.
			\choice Равна $-4 / \sqrt{9\pi}$.
			\choice Равна значению максимума априорной плотности.
			\choice Нет верного ответа.
		\end{choices}
	
		\question Выражение $\P(\mu \in (c, d) | X_{99}) = 0.95$
		\begin{choices}
			\choice Не имеет смысла в байесовском подходе.
			\choice Всегда совпадает с частотным аналогом.
			\choice Эквивалентно выражению $\P(\mu \in (c, d) | X_{99}) > 0.95$.
			\choice Может быть рассчитано, только если $\mu$ -- случайная величина.
			\choice Нет верного ответа.
		\end{choices}
	
		\question Точечная байесовская оценка $\mu$
		\begin{choices}
			\choice Может быть средним, модой или дисперсией апостериорного распределения.
			\choice Не вычисляется аналитически.
			\choice Не существует, если максимум второй производной правдоподобия равен 0 хотя бы в одной точке.
			\choice Вычисляется на основе апостериорного распределения.
			\choice Нет верного ответа.
		\end{choices}
		\vspace{1em}
	
		Далее будем рассуждать в терминах частотного подхода и считать, что $\mu$ -- константа.
		\vspace{1em}
		
		\question Пусть тестируется гипотеза $H_0: \mu = -1$ против $H_1: \mu < -1$. Тогда
		\begin{choices}
			\choice Геометрически p-value является наклоном плотности.
			\choice Если p-value окажется близким к 0.99, то нулевая гипотеза не будет отвергнута на уровне значимости 0.97.
			\choice При использовании LM-теста p-value обязательно окажется близким к 0.
			\choice Если при использовании Z-теста p-value окажется 0.99, то $H_0$ будет отвергнута на любом разумном уровне значимости.
			\choice Нет верного ответа.
		\end{choices}
	
		\question Пусть тестируется гипотеза $H_0: \mu = 1$ против $H_1: \mu \ne 1$. Тогда
		\begin{choices}
			\choice p-value приблизительно равно 0.01.
			\choice Если p-value окажется равным 0.000, то нулевая гипотеза не будет отвергнута на уровне значимости 1\%.
			\choice Для проверки гипотезы достаточно рассчитать лишь p-value / 2.
			\choice Если p-value окажется равным 0.01, то нулевая гипотеза не отвергается на уровне значимости 99\%.
			\choice Нет верного ответа.
		\end{choices}
	\end{questions}
	
	При тестировании трёх видов лекарств против плацебо ($H_{0,i}: p_i = p_{plac}$) оказалось, что соответствующие p-value равны 0.000, 0.99, 0.15.
	
	\begin{questions}
		\question На основании условия задачи можно сделать вывод, что на уровне значимости 5\%
		\begin{choices}
			\item Только одно лекарство статистически отлично от плацебо.
			\item Первое лекарство статистически неотличимо от плацебо на уровне значимости $5\%$.
			\item Все лекарства статистически неотличимы от плацебо.
			\item Только первые два лекарства статистически неотличимы от плацебо.
			\item Нет верного ответа.
		\end{choices}
	
		\question При проведении множественного тестирования методом Бонферрони на уровне значимости 5\%
		\begin{choices}
			\item Только первое лекарство статистически неотличимо от плацебо.
			\item Только второе лекарство статистически отлично от плацебо.
			\item Пороговое значение для отвержения гипотезы следует принять равным $\alpha / 2$.
			\item Не существует разумного уровня значимости, при котором третье лекарство было бы отлично от плацебо.
			\item Нет верного ответа.
		\end{choices}
	
		\question При проведении множественного тестирования методом Бенджамини-Хохберга на уровне значимости 5\%
		\begin{choices}
			\item Невозможно сказать, статистически отлично ли первое лекарство от плацебо.
			\item Второе лекарство окажется статистически отличным от плацебо.
			\item Пороговое значение равно 50\%.
			\item Результаты будут отличны от результатов метода Бонферрони.
			\item Нет верного ответа.
		\end{choices}
	
		\question При проведении множественного тестирования методов Бенджамини-Хохберга на уровне значимости 5\%
		\begin{choices}
			\item Ровно два лекарства статистически неотличимы от плацебо.
			\item Ровно одно лекарство статистически неотличимо от плацебо.
			\item Все три лекарства статистически неотличимы от плацебо.
			\item Все три лекарства статистически отличны от плацебо.
			\item Нет верного ответа.
		\end{choices}
	\end{questions}
	


\end{document} 
	