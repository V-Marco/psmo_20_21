\documentclass[10pt, a4paper]{extarticle}

%% Язык
\usepackage{cmap} % Поиск в PDF
\usepackage{mathtext} % Кириллица в формулах
\usepackage[T2A]{fontenc} % Кодировка
\usepackage[utf8]{inputenc} % Кодировка
\usepackage[english,russian]{babel} % Локализация, переносы

%% Шрифты

% Serif
%\usepackage{euscript} % Шрифт Евклид
%\usepackage{mathrsfs} % Шрифт для математики
\usepackage{libertinus}

% Sans-serif
%\renewcommand{\rmdefault}{cmss}
%\renewcommand{\ttdefault}{cmss}
%\usepackage{sfmath}

% Настройки для xelatex
%\usepackage{polyglossia} % Для выбора языка в xelatex
%\setmainlanguage{russian}
%\setotherlanguages{english}
% Ligatures=TeX is on by default
% https://tex.stackexchange.com/questions/323542/
%\setmainfont[Ligatures=TeX]{Cantarell}
%\newfontfamily{\cyrillicfonttt}{Times New Roman}
%\newfontfamily\cyrillicfont{Cantarell}[Script=Cyrillic]
%\setsansfont[Ligatures=TeX]{Cantarell}
%\newfontfamily\cyrillicfontsf{Cantarell}[Script=Cyrillic]
%\setmonofont{Courier New}
%\newfontfamily\cyrillicfonttt{Courier New}[Script=Cyrillic]

%% Математика
\usepackage{amsmath, amsfonts, amssymb, amsthm, mathtools}
\usepackage{icomma}

% Операторы
\DeclareMathOperator*\plim{plim}
\DeclareMathOperator{\sgn}{sign}
\DeclareMathOperator{\sign}{sign}
\DeclareMathOperator*{\argmin}{arg\,min}
\DeclareMathOperator*{\argmax}{arg\,max}
\DeclareMathOperator*{\amn}{arg\,min}
\DeclareMathOperator*{\amx}{arg\,max}
\DeclareMathOperator{\cov}{Cov}
\DeclareMathOperator{\Var}{Var}
\DeclareMathOperator{\Cov}{Cov}
\DeclareMathOperator{\Corr}{Corr}
\DeclareMathOperator{\pCorr}{pCorr}
\DeclareMathOperator{\E}{\mathbb{E}}
\let\P\relax
\DeclareMathOperator{\P}{\mathbb{P}}
\renewcommand{\le}{\leqslant}
\renewcommand{\ge}{\geqslant}
\renewcommand{\leq}{\leqslant}
\renewcommand{\geq}{\geqslant}

% Распределения
\newcommand{\cN}{\mathcal{N}}
\newcommand{\cU}{\mathcal{U}}
\newcommand{\cBinom}{\mathcal{Binom}}
\newcommand{\cPois}{\mathcal{Pois}}
\newcommand{\cBeta}{\mathcal{Beta}}
\newcommand{\cGamma}{\mathcal{Gamma}}

% Множества
\def \R{\mathbb{R}}
\def \N{\mathbb{N}}
\def \Z{\mathbb{Z}}

% Другое
\newcommand{\dx}[1]{\,\mathrm{d}#1} % Для интеграла: маленький отступ и прямая d
\newcommand{\ind}[1]{\mathbbm{1}_{\{#1\}}} % Индикатор события
\newcommand{\iid}{\mathrel{\stackrel{\rm i.\,i.\,d.}\sim}}
\newcommand{\const}{\mathrm{const}}

%% Изображения
\usepackage{graphicx}
\usepackage{caption}
\usepackage{subcaption}
\usepackage{physics}
\usepackage{wrapfig} % Обтекание рисунков и таблиц текстом
\usepackage{tikz}

%% Таблицы
\usepackage{array, tabularx, tabulary, booktabs}
\usepackage{longtable}  % Длинные таблицы
\usepackage{multirow} % Слияние строк в таблице

%% Cписки
\usepackage{multicol}
\usepackage{enumitem}

%% Гиперссылки
\usepackage{xcolor}
\usepackage{hyperref}
\definecolor{linkcolor}{HTML}{8b00ff}
\hypersetup{colorlinks = true,
			linkcolor = linkcolor,
			urlcolor = linkcolor,
			citecolor = linkcolor}

%% Выравнивание
\setlength{\parskip}{0.5em} % Расстояние между абзацами
\usepackage{geometry} % Поля
\geometry{
	a4paper,
	left=20mm,
	top=20mm,
	right=20mm}
\setlength{\parindent}{0cm} % Отступ (красная строка)
\linespread{1.0} % Интерлиньяж
\usepackage[many]{tcolorbox}  

%% Оформление

\newtcolorbox{rulesbox}[1]{%
	tikznode boxed title,
	enhanced,
	arc=0mm,
	interior style={white},
	attach boxed title to top center= {yshift=-\tcboxedtitleheight/2},
	fonttitle=\bfseries,
	colbacktitle=white,coltitle=black,
	boxed title style={size=normal,colframe=white,boxrule=0pt},
	title={#1}}

% Красивый серый фон
\usepackage{framed} 
\definecolor{shadecolor}{gray}{0.9}

% Код
\newcommand{\code}[1]{{\tt #1}}

% Колонтитулы
\usepackage{fancyhdr}
\pagestyle{fancy}
\fancyhf{}
\fancyhead[L]{}
\fancyhead[R]{\thepage}

% Разделы и подразделы
\usepackage[sf, sl, outermarks]{titlesec}
\titleformat{\section}{\Large\bfseries\sffamily}{\thesection}{0.5em}{}
\titleformat{\subsection}{\large\sffamily}{\thesubsection}{0.5em}{}

% Содержание
%\usepackage{tocloft}
%\renewcommand{\cftsecfont}{\hspace{4.5em}\normalfont}
%\renewcommand{\cftsubsecfont}{\hspace{5em}\normalfont}
%\renewcommand{\cftsecpagefont}{\normalfont\hfill}
%\renewcommand{\cfttoctitlefont}{\large\normalfont\hfill}
%\renewcommand{\cftaftertoctitle}{\hfill}
%\renewcommand{\cftsecleader}{\cftdotfill{\cftdotsep}}
%\renewcommand{\cftsecafterpnum}{\hspace*{5.5em}\hfill}
%\renewcommand{\cftsubsecafterpnum}{\hspace*{5.5em}\hfill}
%\renewcommand{\cftsecaftersnum}{.}
%\renewcommand{\cftsubsecaftersnum}{.}

%% Комментарии
\usepackage{comment}

%% To-do
\usepackage{todonotes}

%% Литература
\usepackage[backend = biber,
			bibencoding = utf8, 
			sorting = nty, 
			maxcitenames = 4,
			style = numeric-verb]{biblatex}
\addbibresource{lit.bib}
\usepackage{csquotes}

%% Заголовок
\title{{\normalsize Прикладная статистика в машинном обучении} \\ \vspace{0.5em} Контрольная работа}
\author{\rule{15cm}{0.4pt}}




\begin{document}
	
	\maketitle

	{\Large \textbf{Задача 1.}}
	
	Исследователь Лосяш хочет оценить факторы, влияющие на богатство жителей маленькой, но гордой планеты Плутон. Известно, что каждый житель имеет Документ, в котором точно указаны его уровень мастерства и уровень начитанности (оба -- вещественнозначные переменные). Исследователь Лосяш строит следующую модель:
	\[
	y_i = \beta_0e^{\beta_1x_i + \beta_2z_i}u_i,
	\]
	где $y_i$ -- богатство $i$-го жителя, $x_i$ -- его уровень мастерства, $z_i$ -- его уровень начитанности. Лосяш уверен, что $\ln u_i \sim \mathcal{N}(0, 1)$.
	
	\begin{enumerate}[label = \alph*)]
		\item Является ли модель линейной по $\beta_2$? А по $\beta_1$? А по $\beta_0$? 
		\item Найди ML-оценки коэффициентов $\beta_1$ и $\beta_2$ при помощи метода максимального правдоподобия.
		\item Выведи формулу $95\%$-го доверительного интервала для $\beta_0$.
		\item Перепиши модель так, чтобы её можно было оценить при помощи МНК. Интуитивно поясни, что нужно предпринять, чтобы МНК-оценки совпадали с ML-оценками.
		\item Напиши подробный алгоритм, как можно проверить гипотезу
		\[
		H_0: \beta_1 = 2\beta_2
		\]
		при помощи LR-теста и теста Вальда.
	\end{enumerate}
	\vspace{1em}
	
	{\Large \textbf{Задача 2.}}
	
	Рассмотрим модель множественной регрессии $y = X\beta + u$, оцениваемую при помощи МНК по 7 регрессорам, включая константный. Число наблюдений равно $n = 200$. 
	
	\begin{enumerate}[label = \alph*)]
		\item Бараш убеждён, что верна только полная модель, построенная на всех 7 регрессорах. Покажи на картинке МНК $\hat{y}$, $TSS$, $ESS$, $RSS$, $R^2$ в его регрессии.
		\item Нюша считает, что полная модель, конечно, верна, но при этом верна и ограниченная модель с пятью регрессорами, включая константный. Покажи на картинке МНК из предыдущего пункта $\hat{y}$, $TSS$, $ESS$, $RSS$, $R^2$ в её регрессии.
		\item Как проверить, права ли Нюша? Покажи на картинке МНК объект, который можно использовать в качестве критерия. 
	\end{enumerate}
	\vspace{1em}
	
	\textit{На следующей страничке ещё две задачи!}
	
	\newpage

	{\Large \textbf{Задача 3.}}

	Исходная выборка $y$ — вектор из $n$ независимых случайных величин, 
	равномерных на $[0;1]$. Пусть $y^*$ — одна из бутстэп-выборок.
	
	\begin{enumerate}[label = \alph*)]
		\item Просто для удобства выпиши $\E(y_i)$, $\Var(y_i)$, $\E(\bar y)$, $\Var(\bar y)$.
		\item Найди $\E(y^*_i)$, $\Var(y^*_i)$, $\E(\bar y^*)$, $\Var(\bar y^*)$.
		\item Найди $\Cov(y_i, y_i^*)$, $\Cov(\bar y, \bar y^*)$.
	\end{enumerate}
	\vspace{1em}
	
	{\Large \textbf{Задача 4.}}
	
	У меня есть три монетки. 
	Они выпадают орлом с вероятностями $p_1$, $p_2$ и $p_3 = p_1 + p_2$.  
	Я провожу эксперимент из $100$ раундов.
	
	В каждом раунде я равновероятно выбираю одну из монеток. 
	Подбрасываю её два раза и записываю число выпавших орлов. 
	
	После окончания эксперимента у меня остаётся на бумажке $100$ записанных чисел. 
	Какая монетка подкидывалась в каждом раунде, я не помню. 
	
	Опиши EM-алгоритм для оценивания неизвестных $p_1$ и $p_2$.
	
	Если формулы для какого-то шага выводятся в явном виде, то выведи их.
	Если формулы для какого-то шага не выводятся в явном виде, то объясни, 
	какая оптимизационная задача будет решаться численно.
	
\end{document}